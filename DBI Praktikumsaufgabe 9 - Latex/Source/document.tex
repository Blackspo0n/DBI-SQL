\documentclass[
	toc=listof, 
	toc=bibliography, % Literaturverzeichnis in das Inhaltsverzeichnis aufnehmen
	footnotes=multiple, % Trennen von direkt aufeinander folgenden Fußnoten
	parskip=half, % vertikalen Abstand zwischen Absätzen verwenden anstatt
	% horizontale Einrückung von Folgeabsätzen
	headlines=1
]{scrartcl}

\usepackage[utf8]{inputenc}
\usepackage[scaled]{uarial}
\usepackage{accsupp}    

\newcommand{\noncopynumber}[1]{
    \BeginAccSupp{method=escape,ActualText={}}
    #1
    \EndAccSupp{}
}

\newcommand{\titel}{Datenbank- und Informationssysteme}
\newcommand{\untertitel}{Praktikumsaufgabe 9 - Transaktionslast}
 
\newcommand{\abgabedatum}{22. Dezember 2016}
\newcommand{\keywords}{DBI, JDBC, SQL, Query, postgresql, Java}
\newcommand{\datum}{\today}
 
\newcommand{\autoren}{
	Jonas Stadtler//jonas.stadler@studmail.w-hs.de,
	Mario Kellner//mario.kellner@studmail.w-hs.de,
	Markus Hausmann//markus.hausmann@studmail.w-hs.de
}

\input{Meta.tex}
\input{Allgemein/Packages.tex} 
\input{Allgemein/Seitenstil.tex}
\input{Allgemein/Befehle.tex}

\begin{document}
\input{Deckblatt.tex}
\clearpage

\pdfbookmark[1]{Inhaltsverzeichnis}{toc}
\pagenumbering{Roman}
\tableofcontents 
\clearpage

\pagenumbering{arabic}
\section{Ausgangssituation}

Aufgrund der Aufgabenstellung ist die Möglichkeit einer Verbesserung an
lediglich drei Stellen gegeben. So könnte man das Absenden der Statements bzw.
deren Inhalt, die zufällige Auswahl der Daten innerhalb der Statements oder die
internen Einstellungen der Datenbank optimieren, um einen höheren Durchsatz zu
erzielen.

Der restliche Ablauf des Programms ist aufgrund der vorgegebene \gqq{ThinkTime}
nicht änderbar oder wird vor dem Ausführen der Messungen durchgeführt. Die Vorgaben machen das
Programm zudem beabsichtigt langsamer als es sein müsste. Denn nach jeder
Abfrage bzw. Update wird eine Pause von 50ms eingelegt, in welcher das Programm
abwartet und keine Transaktionen starten darf. Zudem muss über alle Werte und
Operationen per Zufall entschieden werden. Aufgrund der Vorgaben konvergiert
die maximale Abfrage je Sekunde, bei fünf nebenläufigen Load-Driver-Programmen,
gegen 20.

Bei der Anfrageoptimierung ist es wichtig, möglichst wenig Daten abzufragen, um
unnötigen Datentransfer zu verhindern. Zudem sollte darauf geachtet werden
möglichst wenige Abfragen durchzuführen, dies kann zum Beispiel durch das
Verwenden von Subselects erreicht werden. Um die Abfrage noch weiter zu
verbessern ist es außerdem wichtig die Datenbanksprache SQL gut zu kennen und
möglichst viele Operationen von dieser bearbeiten zu lassen. Dies spart nicht
nur unnötigen Code ein, sondern spart auch Abfragen und Datentransfer.

Eine weitere Möglichkeit zur Optimierung des Programms ist es, die Daten welche
zufällig für die Abfragen ausgewählt werden nicht über die Funktion
MATH.Random() zu bestimmen. Das Gleiche gilt auch für die verteilte Auswahl der
auszuführenden Operationen. Ein Lösungsansatz wäre es, die einzutragenden Daten
vor dem Start des Benchmarkings in ein Array oder .txt Datei einzutragen und
diese dann lediglich an den entsprechenden Stellen einzutragen. Für die Auswahl
der Operation gilt das Gleiche Prinzip. So kann man vor dem Start des
Benchmarkings ein Array mit Zahlen von eins bis 100 Füllen. Bei der Abfrage in
der Schleife können dann die Zahlen eins bis 35 für die Auswahl des
Kontostandes verwendet werden, 36 bis 85 für die Auswahl Einzahlung und 86 bis
100 für die Auswahl des Kontostandes stehen. Dabei ist die Voraussetzung eines
zufällig Ausgewählten Vorgangs in der zehnminütigen Schleife des
Load-Driver-Programms erfüllt. Denn das Programm verwendet zufällige Werte,
welche die geforderten Wahrscheinlichkeiten erfüllen. Zudem erfolgt die
Auswertung, der Werte und die Entscheidung welche Funktion gestartet wird
innerhalb der Schleife.

Die dritte Möglichkeit ist es, die Einstellungen der Datenbank zu Optimieren.
Diese Optimierung wirkt sich auf die Ressourcennutzung, sowie die
Nachverfolgbarkeit der Datenbankeinträge bzw. Abfragen aus. Dabei sind die
technischen Voraussetzungen, welche der verwendete Computer bietet von großer
Bedeutung. Besonders wichtig ist dabei die Art der Festplatte, so ist eine
SSD-Festplatte in der Lage deutlich schneller Daten zu schreiben oder zu lesen,
als eine HDD-Festplatte. Aber nicht nur die Festplatte ist von Bedeutung, denn
ein Puffer, welcher beim Schreiben besonders wichtig ist, benötigt
Arbeitsspeicher. Des Weiteren, ist ein s tarker Prozessor von Vorteil, da dieser
ein schnelleres Ausführen des Programms ermöglicht.

\clearpage
\section{Implementierung der Aufgabe}

\subsection{Verbindungsaufbau}
Der erste Schritt beim Erstellen des Programms befasste sich mit dem Erstellen
einer Verbindung zur PostgreSQl-Datenbank. 

Dazu wurde die Klasse \textbf{ConnectionInformation}, sowie
\textbf{ConnectionInformation} aus dem vorherigen Projekt implementiert. 

Die Klasse \textbf{ConnectionInformation} dient der Erstellung eines gleichnamigen
Objektes, welches alle Informationen zur Datenbankanbindung enthält und an das
Objekt vom Typ \textbf{DatabaseConnection} übergeben wird. Mit Hilfe der Funktion
\textbf{connect()}, des Objekts \textbf{DatabaseConnection}, wird eine Verbindung
erzeugt, welche die übergebenen Parameter verwendet.

\subsection{Erstellen der Abfragen}
Die Erstellung der Abfragen findet in der Klasse \textbf{TXHandler} statt, welche
ein gleichnamiges Objekt erzeugt. 

Das entstehende Objekt kann die Funktion \gqq{kontostandTX} aufrufen, welche den
Kontostand abfragt und zurückgibt.Es kann außerdem mit Hilfe der Funktion
einzahlungTX eine Einzahlung vornehmen, welche den aktuellen Kontostand zurückgibt.


---------------- code hier einfügen

Des Weiteren kann mit der Funktion \textbf{analyseTX} nach der Häufigkeit der
Einzahlung eines bestimmten Betrags gesucht werden, wobei die Häufigkeit zurückgegeben wird.

\subsection{Erstellung der Programm-Schritte}
Die oben beschriebenen Abfrage-Möglichkeiten werden in den vorgegebenen
Schritten Einschwingphase, Messphase und Ausschwingphase in einer Gwichtung von
35/50/15 ausgeführt.

Die Verteilung nach diesem Verhältnis übernimmt dabei die Funktion
\gqq{choose()}.

--- code verteilungsfunktion

Das Aufrufen der jeweiligen Funktion mit zufällig gewählten Parametern übernimmt die Funktion
doTX(). Wobei die zufälligen Werte durch die Funktion chooseNumber() bestimmt
werden, diese Funktion achtet dabei darauf, dass keine unzulässigen Werte
abgefragt werden.


-- code doTX

Die Einschwingphase wird in der Funktion \textbf{attackStage()} implementiert
und dauert wie in den Vorgaben beschrieben genau vier Minuten.

Die Messphase wird als Funktion \textbf{benchStage()} implementiert und dauert
genau fünf Minuten.

Die Ausschwingphase wird in der Funktion \textbf{boomOutStage()} dargestellt und
dauert gemäß der Anforderungen eine Minute.

In jeder Funktion wird die \gqq{thinkTime} von 50 Millisekunden berücksichtigt.

\subsection{Erstellung der Threads}

Die Threads, welche jeweils einen Lastclient bereitstellt wird in der Klass
\gqq{ClientThread} implementiert. Diese erbt von der Klasse Thread und
implemtiert ein \textbf{run}-Funktion.

----- code ausschnitt clientthread

Die Instanziierung der Threads finden anschließend in der
\textbf{main}-Methode des Programms statt. Nach der Erstellung der Threads
soll mittels eines \textbf{timestamps} auf ein Startsignal gewartet werden.

Das Startsignal wird gegeben, wenn der zuvor festgelegte \textbf{timestamp}
überschritten worden ist:

----------- protgrammcode von der schleife


\subsection{Exception-Handling}

Das Exceptionhandlung findet global in der \textbf{main}-Methode statt. Tritt
ein Fehler auf, wird dieser mittels dem \textbf{throws}-Statement
\gqq{hochgebubbelt} bis dieser in unserer \textbf{catch}-Klausel behandelt wird.

---- code vom catchblock

Da die \textbf{run}-Methode der Klasse \textbf{Thread} kein \textbf{throws}
unterstützt würde die fehlerbehandlung für die Threads gesondert implementiert.
Zusätzlich wird von den Threads die Funktion \textbf{abortProgramm}
ausgeführt, wenn ein unbehandelter fehler im Thread auftreten sollte. Dies führt
dazu, dass dsa Programm automatisch die Ausführung des Programmcodes
unterbricht.

\clearpage
\section{Optimierungen im Programm}
Das erstellte Programm war bereits in der ersten Version effizient. Denn die
benötigten SQL-Befehle waren von Anfang an optimiert. Bei der Erstellung der
SQL-Befehle wurde darauf geachtet möglichst wenige Befehle mit möglichst wenig
abgefragten Variablen  zu erstellen.

(Quelltext, wo die SQL Befehle gezeigt werden)

Durch das Verwenden des Operators count (*) sparen wir uns eine Schleife im
Programm, welche die Anzahl der Elemente, mit der geforderten Bedingung,
ermittelt. Dabei wird auch Zeit gespart, da sich lediglich ein Element im
ResultSet befindet und auch nur eins aufgerufen werden muss. Eine weitere
Optimierung ist die Verwendung eines Subselects beim Insert-Befehl. Dieser
ermöglicht es auf einen Select-Befehl verzichten zu können.

Die Verwendung von einfachen excecuteUpdate-Befehlen macht in diesem Fall mehr
Sinn als das Verwenden von PreparedStatements. Denn wir müssten bei der
Verwendung von PreparedStatements für jede einzelne Tabelle, welche wir
anspreche wollen, ein neues PreparedStatements erstellen um ein Update
durchzuführen. Dadurch würden wir keine höhere Effizienz erreichen.

Die einzig effiziente Änderung die wir durchgeführt haben ist, dass wir die
Zufallswerte innerhalb der 50ms Pause durchgeführt wird und wir dadurch je
Durchgang einige tausendstel Sekunden sparen.

\clearpage
\section{Änderungen am Datenbank Management System}

Am Datenbankmanagement-System (DBMS) sind einige Einstellungen über die
postresql.conf änderbar. In dieser Datei sind die Einstellungen des DBMS
gespeichert und können vom Benutzer editiert werden.

Zunächst wird von uns die Anzahl der Verbindungen, welche gleichzeitig zur
Datenbank bestehen dürfen (\textbf{max\_connections}) von voreingestellten
\textbf{100} auf die von uns benötigten \textbf{18} heruntergesetzt. 
Dies soll einen Performancegewinn bringen, da zu viele erlaubte Verbindungen
unnötig Ressourcen ziehen. Zudem haben wir eine ideale Ausgangssituation, da auf
8 virtuellen Kernen lediglich 18 Verbindungen existieren. Dies erfüllt die
Idealbedingung von zwei bis drei erlaubten Verbindungen je Kern.

Wir haben uns im Weiteren mit der Pufferspeicherung in der .conf Datei befasst,
welche \textbf{shared\_buffers} heißt. Dazu wird dem Pufferspeicher ¼ des Rams
zugewiesen. Der restliche Ram wird für den \textbf{effective\_cache\_size}
verwendet, welcher der Planung und Ausführung von Querys dient.

Zudem haben wir dem \textbf{work\_mem} einen höheren Wert zugewiesen, eine
Optimierung ist hierbei nur möglich, wenn man die Anzahl der maximalen Verbindungen kennt.


Wir haben uns für den Wert $RAM/(max\_connections * 16)$ entschieden, da
dieser sich aufgrund von Internetrecherche und nach einigen Tests als effektiv
erwiesen hat. \textbf{work\_mem }ist für die Verknüpfung von Tabellen oder
Umsetzung bestimmter Klauseln wichtig und weist zu, wie viel Speicher je Sortier-und
Hashoperation verwendet werden kann.

\clearpage
\section{Ausführung der Messung}

Zum Ausführen von 5 remote Load Driver wird zunächst für alle 5 erzeugten
Threads, welche die Nebenläufigkeit ermöglichen, eine Verbindung zur Datenbank
erzeugt. Diese Verbindung zur Datenbank wird zunächst für das leeren der
Tabelle history verwendet. Nach dem leeren der Tabelle beginnen die Threads
gleichzeitig mit der Einschwingphase und gehen nach dieser in die Messphase
über, um nach der folgenden Ausschwingphase das Programm zu beenden.

Mit dem Start der Einschwingphase beginnen die Threads mit der zufälligen
Auswahl einer Abfrage an die Datenbank, dies erfolgt unabhängig voneinander.
Nach der Auswahl der Abfrage werden die Zufallsdaten für die Abfrage mit Hilfe
der Funktion chooseNumber() ermittelt, welche die Informationen über die
benötigt Variable übermittelt bekommt. Durch das übergeben der Bezeichnung, der
benötigten Variablen, und der Größe der Datenbank, soll gewährleistet sein,
dass es zu keinen fehlerhaften Abfragen und einer dadurch fehlerhaften Messung
kommt. Die Zahlen werden von der Funktion im zulässigen Bereich zufällig
ausgewählt und sind unabhängig von den nebenläufigen Threads. Nach dem
Ausführen der Anfrage wird eine Pause von 50ms eingelegt. In der Messphase wird
vor der Pause eine Variable um einen erhöht, welche die Anzahl der Abfragen
zählt und im weiteren Verlauf, mit Hilfe von synchronized, die Anzahl aller
Abfragen aus den 5 Threads, in der Messphase, addiert. Diese Variable ist die
Grundlage für die Rechnung der Anfragen je Sekunde.

Bei der Ausführung auf mehreren Computern werden die Threads zu einem
vordefinierten Zeitpunkt,  gleichzeitig auf allen Computern, gestartet. Die
Ermittlung der Anfragen je Sekunde erfolgt nach Beendigung der Programme
händisch, durch das Addieren aller Anfragen und das Teilen dieser durch die 5
Minuten bzw. 300 Sekunden der Messphase.

\clearpage
\section{Bewertung und Dokumentation der Lasten}

\subsection{Last 5}
Bei der ersten Last, also der Verwendung von 5 nebenläufigen
Load-Driver-Programmen auf einem PC, haben wir durch die Optimierungen eine
Verbesserung von 85 TPS auf insgesamt 95 erreicht. Nach der Verbesserung des
Clients, also der Ermittlung der Zufallsvariablen während der Pause von 50ms,
erreichten wir einen Wert von 93 TPS. Bei einigen anderen Messung erreichten
wir lediglich geringe Ergebnisse, da nebenher laufende Programme am
Test-Computer die Messergebnisse negativ beeinflusst haben.

Nach der Optimierung des Programms haben wir das DBMS optimiert und einen Wert
von 95 TPS erreicht. Dieser Wert ist stabil, weicht also bei weiteren Messungen
maximal um 0,5 TPS nach Unten oder Oben ab.


\subsection{Last 5x5}
Bei der zweiten Last, also der Verwendung von 5 nebenläufigen
Load-Driver-Programmen auf zwei Computern, welche Abfragen an die gleiche
Datenbank schicken, haben wir einen Bestwert von 185 TPS erzielt. Bei der
ersten Messung, also der Messung des nicht optimierten Programms, haben wir
einen Wert von 243 TPS erreicht, welchen wir durch die Optimierung des
Programms auf 181 TPS erhöhen konnten. Trotz Optimierung der Datenbank war es
nicht möglich, einen Wert über 185 TPS zu erreichen.

Dies ist für uns nicht nachvollziehbar, denn wir haben durch das Erhöhen des
Puffers und einer Anpassung der Einstellung auf die benötigte Funktionalität,
eine ideale Situation für die Abfrage und das Schreiben von Daten geschaffen.
Besonders bei einer größeren Menge von Anfragen haben wir mit einer deutlich
stärkeren Leistung des DBMS gerechnet, wenn man die Einstellungen optimiert.

Alle Ergebnisse, in jeder Phase der Entwicklung, waren stabil und es gab auch
keine Probleme mit dem Einfluss von nebenher laufenden Programmen. Jedoch ist
uns aufgefallen, dass immer ein Computer bessere Ergebnisse liefert als der
Andere. Daher vermuten wir, dass die Abfragen, je nachdem welcher Computer diese
abschickt, priorisiert oder zunächst zurückgestellt werden.

\clearpage

\subsection{Last 5x5x5}
Bei der dritten Last, also der Verwendung von 5 nebenläufigen
Load-Driver-Programmen auf drei Computern, welche Abfragen an die gleiche
Datenbank schicken, haben wir einen Bestwert von 267 TPS erzielt. Bei unserer
ersten Messung haben wir lediglich einen Wert 243 TPS erreicht, welchen wir
durch die Verbesserung des Programms um 9 TPS, auf 252 TPS, steigern konnten.
Diese Steigerung  ist im Verhältnis zu den beiden anderen Lasten eher gering.
Denn hier steigern wir die Leistungsfähigkeit des Programms und somit müsste
die Leistung um den dreifachen Wert steigen, als sie
dies bei Last eins, in der gleichen Situation, getan hat. Demnach müsste die
Anzahl der TPS statt um 9, um etwa 3 mal 8, also um etwa 24 steigen.

Die Steigerung von 252 TPS auf 267, bei der Optimierung des DBMS ist ebenfalls
geringer als erwartet. Denn unserer Überlegung nach müssten sich die
Auswirkungen der Optimierung besonders stark bei einer größeren Menge von
Abfragen zeigen. Dem ist aber folglich nicht so, dies könnte daran liegen, dass
das DBMS mit einer solchen Menge an Abfragen und den bereitgestellten
Ressourcen der virtuellen Maschine nicht auskommt und lediglich eine Erhöhung
der Ressourcen zu einem besseren Ergebnis führt.

Bei dieser Last war es uns leider nicht möglich so viele Messungen wie bei den
Anderen durchzuführen. Somit ist es nicht möglich Schwankungen der Ergebnisse
auszuschließen, jedoch sind uns keine besonders starken Abweichungen
aufgefallen.

\clearpage
\section{Fazit zu PostgreSQL}


\clearpage

\clearpage
 
\lstlistoflistings
\clearpage

\appendix
\pagenumbering{roman}
\section{Anhang}
\section{Anhang}

\begin{lstlisting}[caption={Main}, label={lst:pr1}]
package de.w_hs.dbi.pa9;

import java.sql.SQLException;

import de.w_hs.dbi.pa9.database.ConnectionInformation;
import de.w_hs.dbi.pa9.database.DatabaseConnection;
import de.w_hs.dbi.pa9.function.ProgramStage;
import de.w_hs.dbi.pa9.function.TXHandler;

public class Main 
{
	public static DatabaseConnection connection=null;
	public static int txCountSum;
	

	public static synchronized int getTxCountSum()
	{
		return txCountSum;
	}

	public static synchronized void setTxCountSum(int txCountSum)
	{
		Main.txCountSum = txCountSum;
	}

	public static long time = 1482235928208L;
	
	/**
	 * Main Funktion.
	 * @param args
	 * @throws Exception
	 * @author Mario Kellner
	 * @author Markus Hausmann
	 * @author Jonas Stadtler
	 */
	public static void main(String args[])
	{
		ConnectionInformation infos = new ConnectionInformation();
		infos.setHost("192.168.122.70");
		infos.setDatabase("benchmark");
		infos.setUser("postgres");
		infos.setPassword("DBIPr");
		
	
		try
		{
			// Erstelle Verbindung um die History !einmalig! zu löschen!			
			DatabaseConnection con = new DatabaseConnection(infos);
			con.connect();
			con.clearHistory();
			
		    // Create the threads
		    Thread[] threadList = new Thread[5];

		    // spawn threads
		    for (int i = 0; i < 5; i++)
		    {
		        threadList[i] = new ClientThread(infos);
		        threadList[i].start();
		    }
		    
		    // Start everyone at the same time
		    System.out.println("Wartezeit für die Threads beträgt " + (time - System.currentTimeMillis()) + " Millisekunden");
		    while(time > System.currentTimeMillis())
		    {
		    	Thread.sleep(1);
		    }
		    
		    System.out.println("Wartezeit erreicht");
		    
		    ClientThread.setStartTrans();
			
			for (int i = 0; i < 5; i++)
	        {
	           threadList[i].join();
	        }
		    
			System.out.println("Gesamte Anzahl der Transaktionen: " + getTxCountSum());
			System.out.println("TPS: " + (double)getTxCountSum()/(double)300);
			
			System.out.println("Finish!");
		    
		} 
		catch (SQLException sqle)
		{
			System.out.println("Datenbank Fehler: " + sqle.getMessage());
			sqle.printStackTrace();
		}
		catch (Exception e)
		{
			System.out.println("Ein Fehler ist aufgetreten: " + e.getMessage());
			e.printStackTrace();
		}
	}
	
	public static void abortProgramm() {
		System.exit(0);
	}
}
\end{lstlisting}

\begin{lstlisting}[caption={ClientThread}, label={lst:pr2}]
package de.w_hs.dbi.pa9;

import java.sql.SQLException;

import de.w_hs.dbi.pa9.database.ConnectionInformation;
import de.w_hs.dbi.pa9.database.DatabaseConnection;
import de.w_hs.dbi.pa9.function.ProgramStage;

/**
 *  Zuweisungen an Threads.
 *  
 *  @author Mario Kellner
	@author Markus Hausmann
 *  @author Jonas Stadtler
 *
 */
public class ClientThread extends Thread {
	private DatabaseConnection threadCon;
	static boolean startTrans = false;
	private int threadID;
	private static int nextThreadID = 1;
	
	private synchronized static int getNextId() {
		return nextThreadID++;
	}


	
	public ClientThread(ConnectionInformation infos) throws Exception {
		threadCon = new DatabaseConnection(infos);
		threadID = getNextId();
	}
	
	static synchronized void setStartTrans () {
		startTrans = true;
	}

	synchronized boolean getStartTrans () {
		return startTrans;
	}

	public void run() {
		try {
			threadCon.connect();
			
			if(threadCon.databaseLink.isValid(0))
			{
				System.out.println("[Thread " + threadID +"]Thread erfolgreich zur Datenbank verbunden!");
			}
			else
			{
				throw new SQLException("Keine Verbindung zur Datenbank möglich");
			}
			
			System.out.println("[Thread " + threadID +"]Auf Startsignal warten ...");
			
			while(!getStartTrans()) {
				yield();
			}
			
			System.out.println("[Thread " + threadID +"]Startsignal erhalten!");
			
			
			ProgramStage program = new ProgramStage(threadCon);
			

			System.out.println("[Thread " + threadID +"]Führe Funktion attackStage aus!");
			program.attackStage();
			
			System.out.println("[Thread " + threadID +"]Führe Funktion benchStage aus!");
			program.benchStage();
			
			System.out.println("[Thread " + threadID +"]Führe Funktion boomOutStage aus!");
			program.boomOutStage();
		}
		catch (SQLException sqle)
		{
			System.out.println("Datenbank Fehler: " + sqle.getMessage());
			sqle.printStackTrace();
			
			Main.abortProgramm();
		}
		catch (Exception e)
		{
			System.out.println("Ein Fehler ist aufgetreten: " + e.getMessage());
			e.printStackTrace();

			Main.abortProgramm();
		}
			
	}
}
\end{lstlisting}


\begin{lstlisting}[caption={ProgramStage}, label={lst:pr3}]
package de.w_hs.dbi.pa9.function;

import java.sql.SQLException;

import de.w_hs.dbi.pa9.Main;
import de.w_hs.dbi.pa9.database.DatabaseConnection;

/**
 * 
 * Diese Klasse enthält alle die drei Schritte des Programmablaufs. Zudem werden hier die Abfrage
 * Werte per Zufall definiert und die Gewichtung erfolgt im geforderten Verhältnis.
 *
 * @author Mario Kellner
 * @author Markus Hausmann
 * @author Jonas Stadtler
 * 
 *
 */
public class ProgramStage 
{
	private int TransactionNumber;
	private int AccountsNumber;
	private int BranchesNumber;
	private int TellersNumber;
	private int deltaNumber;
	private DatabaseConnection connection;
	private TXHandler txh;
	
	public ProgramStage(DatabaseConnection con) throws SQLException
	{
		if(con == null)
		{
			throw new NullPointerException("Connection darf nicht NULL sein!");
		}
		
		this.connection = con;
		this.txh = new TXHandler(con);
	}
	
	/**
	 * Entscheidet per gewichteten Zufall darüber, welche Abfrage an die Datenbank geschickt wird..
	 * 
	 * @return Gibt die Zahl 1, 2 oder 3 zurück. Dadurch wird im folgenden über die Abfrage entschieden.
	 */
	public int choose()
	{
		int number=(int)(Math.random()*100);
		if(number <= 35)
		{
			return 1;
		}
		else if( number > 35 && number <= 85)
		{
			return 2;
		}
		return 3;
	}
	
	/**
	 * Entscheidet über die jeweiligen Zahlen, welche mit der Abfrage zur Datenbank geschickt werden.
	 * @param name Bezeichnung der anzusprechenden Tabelle.
	 * @param n Größe der Datenbank in TPS.
	 * @return Zufällige, aber passende Übergabewerte, welche zur Abfrage an die Datenbank 
	 * verwendet werden.
	 */
	private int chooseNumber(String name, int n)
	{
		double number=Math.random();
		switch(name)
		{
			case "accounts":
				number*=10000*n;
				break;
			case "branches":
				number*=n;
				break;
			case "tellers":
				number*=10*n;
				break;
			case "delta":
				number*=10000;
				break;
		}
		return (int)(number+1);
	}
	
	public void thinkTime() throws InterruptedException {
		long startTime = System.currentTimeMillis();
		
		this.generateNumbers();
		
		while(System.currentTimeMillis()-startTime < 50) {
			Thread.sleep(1);
		}
	}
	
	public void generateNumbers() {
		this.TransactionNumber = choose();
		this.AccountsNumber = chooseNumber("accounts", 100);
		this.TellersNumber = chooseNumber("tellers", 100);
		this.BranchesNumber = chooseNumber("branches", 100);
		this.deltaNumber = chooseNumber("delta", 100);
	}
	
	
	/**
	 * Entscheidet über die auszuführende Abfrage. 
	 * Die zufälligen Werte werden von der Funktion generateNumbers berechnet.
	 * 
	 * @throws SQLException Fehler, welcher bei der Abfrage an die Datenbank auftreten kann.
	 */
	private void doTX() throws SQLException
	{
		switch(this.TransactionNumber)
		{
			case 1:
				txh.kontostandTX(this.AccountsNumber);
				break;
			case 2:
				txh.einzahlungTX(
						this.AccountsNumber,
						this.TellersNumber,
						this.BranchesNumber,
						this.deltaNumber
				);
				break;
			case 3:
				txh.analyseTX(this.deltaNumber);
				break;
		}
	}
	
	/**
	 * Ausführen des ersten Programmteils.
	 * 
	 * @throws SQLException Möglicher Fehler bei der DB-Abfrage.
	 * @throws InterruptedException Möglicher Fehler bei der Wartezeit.
	 */
	public void attackStage() throws SQLException, InterruptedException
	{	
		long start = System.currentTimeMillis();
		
		// generiere Initialnummern
		this.generateNumbers();
		
		while(start + 240000 >= System.currentTimeMillis())
		{
			doTX();
			this.thinkTime();
		}
	}
	
	/**
	 * Ausführen des zweiten Programmteils.
	 * 
	 * @throws SQLException Möglicher Fehler bei der DB-Abfrage.
	 * @throws InterruptedException Möglicher Fehler bei der Wartezeit.
	 */
	public void benchStage() throws SQLException, InterruptedException
	{
		int txCounter = 0;
		long start = System.currentTimeMillis();
		
		// generiere Initialnummern
		this.generateNumbers();
		
		while(start + 300000 >= System.currentTimeMillis())
		{
			doTX();
			txCounter++;
			this.thinkTime();	
		}
		
		Main.setTxCountSum(Main.getTxCountSum() + txCounter);
		
		System.out.println("Anzahl der Transaktionen: " + txCounter);
		System.out.println("Anzahl der Transaktionen pro Sekunde: " + (double)((double)txCounter/(double)300));
		
	}
	
	/**
	 * Ausführen des dritten Programmteils.
	 * 
	 * @throws SQLException  Möglicher Fehler bei der DB-Abfrage.
	 * @throws InterruptedException Möglicher Fehler bei der Wartezeit.
	 */
	public void boomOutStage() throws SQLException, InterruptedException
	{
		long start = System.currentTimeMillis();
		
		// generiere Initialnummern
		this.generateNumbers();
		
		while(start + 60000 >= System.currentTimeMillis())
		{
			doTX();
			this.thinkTime();
		}
	}
}

\end{lstlisting}


\begin{lstlisting}[caption={TXHandler}, label={lst:pr4}]
package de.w_hs.dbi.pa9.function;

import java.sql.PreparedStatement;
import java.sql.ResultSet;
import java.sql.SQLException;
import java.sql.Statement;

import org.postgresql.util.PSQLException;

import de.w_hs.dbi.pa9.database.DatabaseConnection;

/**
 * In dieser Klasse sind die Statements hinterlegt, welche an die Datenbank geschickt werden.
 * 
 * @author Mario Kellner
 * @author Markus Hausmann
 * @author Jonas Stadtler
 *
 */
public class TXHandler 
{
	private DatabaseConnection connection;
	private Statement statement;
	
	public TXHandler(DatabaseConnection con) throws SQLException
	{
		this.connection=con;
		statement = connection.databaseLink.createStatement();
	}
	
	/**
	 * Abfrage eines Kontostandes.
	 * 
	 * @param accID Account ID 
	 * @return Kontostand der gesuchten Account ID
	 * @throws SQLException Fehler bei der Datenbank Abfrage, mit spezifizierter Fehlermeldung.
	 */
	public double kontostandTX(int accID) throws SQLException
	{
		Statement statement = connection.databaseLink.createStatement();
		
		ResultSet rs = statement.executeQuery("SELECT balance FROM accounts WHERE accid = " + accID);
		
		if(rs.next() == false)
		{
			throw new SQLException("Kontonummer nicht vorhanden!");
		}
		
		return rs.getDouble(1);
	}
	
	/**
	 * Ermittlung eiens neuen Kontostandes.
	 * 
	 * @return Neuer errechneter Kontostand.
	 * @throws SQLException Fehler bei der Datenbank Abfrage, mit spezifizierter Fehlermeldung.
	 */
	public double einzahlungTX(int accID, int tellerID, int branchID, double delta) throws SQLException
	{
		ResultSet rs = null;
		
		try {
			connection.databaseLink.setAutoCommit(false);
			
			statement.executeUpdate("UPDATE branches SET balance=balance+"+delta+" WHERE branchid="+branchID);
			statement.executeUpdate("UPDATE tellers SET balance=balance+"+delta+" WHERE tellerid="+tellerID);
			statement.executeUpdate("UPDATE accounts SET balance=balance+"+delta+" WHERE accid="+accID);
			statement.executeUpdate("INSERT INTO history (accid, tellerid, delta, branchid, accbalance, cmmnt)"
					+ "VALUES("+accID+", "+tellerID+", "+delta+", "+branchID+", (SELECT balance FROM accounts WHERE accid="+accID+"), ' ')");
			
			rs = statement.executeQuery("SELECT balance FROM accounts WHERE accid = " + accID);
			
			connection.databaseLink.commit();
			
			if(rs.next() == false)
			{
				throw new SQLException("Kontonummer nicht vorhanden");
			}
			
		}
		catch (PSQLException e) {
			System.out.println(e.getMessage());
			connection.databaseLink.rollback();
			
		}
		finally {

			connection.databaseLink.setAutoCommit(true);
		}
		
		return rs.getDouble(1);
	}
	
	/**
	 * Anzahl der Einzahlungen mit dem Betrag delta sollen ermittelt werden.
	 * @param delta Einzahlungsbetrag
	 * @return Anzahl der Einzahlungen mit dem Betrag delta wird zurückgegeben.
	 * @throws SQLException Fehler bei der Datenbank Abfrage, mit spezifizierter Fehlermeldung.
	 */
	public int analyseTX(double delta) throws SQLException
	{
		Statement statement = connection.databaseLink.createStatement();
		
		ResultSet rs = statement.executeQuery("SELECT COUNT(*) FROM history WHERE delta=" + delta);
		
		if(rs.next() == false)
		{
			throw new SQLException("Kein Eintrag mit dem Wert " + delta + " vorhanden!");
		}
		
		return rs.getInt(1);
	}
}
\end{lstlisting}


\begin{lstlisting}[caption={DatabaseConnection}, label={lst:pr5}]
package de.w_hs.dbi.pa9.database;

import java.sql.Connection;
import java.sql.DriverManager;
import java.sql.SQLException;
import java.sql.Statement;

/**
 * Unsere Datebankklasse ermöglicht die Kommunikation
 * mit der Datenbank. Es ist ein Wrapper für das
 * Connection-Objekt der JDBC Connection
 * 
 * @author Mario Kellner
 * @author Markus Hausmann
 * @author Jonas Stadtler
 */
public class DatabaseConnection 
{
	/**
	 * Das eigentliche Connection Objekt
	 */
	public Connection databaseLink;
	
	/**
	 *  Unser Verbindungs-Model
	 */
	private ConnectionInformation ci;
	
	/**
	 * Der Konstruktur der Klasse.  Überprüft ob das Objekt im C
	 * 
	 * @param cInfo Das Connection Objekt
	 * @throws Exception Wird geworfen, wenn das Objekt was übergeben wird null ist.
	 */
	public DatabaseConnection(ConnectionInformation cInfo) throws Exception {
		if(cInfo == null) {
			throw new Exception("ConnectionInfo Object cannot be null");
		}
		
		ci = cInfo;
	}
	public void clearHistory() throws SQLException
	{
		Statement statement= databaseLink.createStatement();

		statement.executeUpdate("TRUNCATE TABLE history");
	}
	/**
	 * 
	 * List die Informatioen aus dem Objekt ci (ConnectionInformation) und versucht eine Verbindung 
	 * zur Datenbank auf zu bauen.
	 * 
	 * @throws SQLException Wird geworfen, wenn der DriverManager keine Verbndung
	 zur Datenbank aufbauen kann */
	public void connect() throws SQLException
	{
		databaseLink = DriverManager.getConnection(
				"jdbc:postgresql://" + ci.getHost() +"/" + ci.getDatabase() + "?useCompression=true",
				ci.getUser(), 
				ci.getPassword()
		);
	}
}
\end{lstlisting}

\begin{lstlisting}[caption={ConnectionInformation}, label={lst:pr6}]
package de.w_hs.dbi.pa9.database;

/**
 * Dieses Model beinhaltet alle nötigen Informationen 
 * um sich zu einer Postgre Datenbank zu verbinden
 * 
 * @author Mario Kellner
 * @author Markus Hausmann
 * @author Jonas Stadtler
 *
 */
public class ConnectionInformation {
	
	// Eigenschaften
	public String host = null;
	public String database = null;
	public String user = null;
	public String password = null;
	
	
	// Getter & Setter
	public String getHost() {
		return host;
	}
	public void setHost(String host) {
		this.host = host;
	}
	public String getDatabase() {
		return database;
	}
	public void setDatabase(String database) {
		this.database = database;
	}
	public String getUser() {
		return user;
	}
	public void setUser(String user) {
		this.user = user;
	}
	public String getPassword() {
		return password;
	}
	public void setPassword(String password) {
		this.password = password;
	}
}
\end{lstlisting}

\clearpage

 
\clearpage
\end{document} 

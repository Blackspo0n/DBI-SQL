\section{Anhang}

\begin{lstlisting}[caption={Main}, label={lst:pr1}]
package de.w_hs.dbi.pa9;

import java.sql.SQLException;

import de.w_hs.dbi.pa9.database.ConnectionInformation;
import de.w_hs.dbi.pa9.database.DatabaseConnection;
import de.w_hs.dbi.pa9.function.ProgramStage;
import de.w_hs.dbi.pa9.function.TXHandler;

public class Main 
{
	public static DatabaseConnection connection=null;
	public static int txCountSum;
	

	public static synchronized int getTxCountSum()
	{
		return txCountSum;
	}

	public static synchronized void setTxCountSum(int txCountSum)
	{
		Main.txCountSum = txCountSum;
	}

	public static long time = 1482235928208L;
	
	/**
	 * Main Funktion.
	 * @param args
	 * @throws Exception
	 * @author Mario Kellner
	 * @author Markus Hausmann
	 * @author Jonas Stadtler
	 */
	public static void main(String args[])
	{
		ConnectionInformation infos = new ConnectionInformation();
		infos.setHost("192.168.122.70");
		infos.setDatabase("benchmark");
		infos.setUser("postgres");
		infos.setPassword("DBIPr");
		
	
		try
		{
			// Erstelle Verbindung um die History !einmalig! zu löschen!			
			DatabaseConnection con = new DatabaseConnection(infos);
			con.connect();
			con.clearHistory();
			
		    // Create the threads
		    Thread[] threadList = new Thread[5];

		    // spawn threads
		    for (int i = 0; i < 5; i++)
		    {
		        threadList[i] = new ClientThread(infos);
		        threadList[i].start();
		    }
		    
		    // Start everyone at the same time
		    System.out.println("Wartezeit für die Threads beträgt " + (time - System.currentTimeMillis()) + " Millisekunden");
		    while(time > System.currentTimeMillis())
		    {
		    	Thread.sleep(1);
		    }
		    
		    System.out.println("Wartezeit erreicht");
		    
		    ClientThread.setStartTrans();
			
			for (int i = 0; i < 5; i++)
	        {
	           threadList[i].join();
	        }
		    
			System.out.println("Gesamte Anzahl der Transaktionen: " + getTxCountSum());
			System.out.println("TPS: " + (double)getTxCountSum()/(double)300);
			
			System.out.println("Finish!");
		    
		} 
		catch (SQLException sqle)
		{
			System.out.println("Datenbank Fehler: " + sqle.getMessage());
			sqle.printStackTrace();
		}
		catch (Exception e)
		{
			System.out.println("Ein Fehler ist aufgetreten: " + e.getMessage());
			e.printStackTrace();
		}
	}
	
	public static void abortProgramm() {
		System.exit(0);
	}
}
\end{lstlisting}

\begin{lstlisting}[caption={ClientThread}, label={lst:pr2}]
package de.w_hs.dbi.pa9;

import java.sql.SQLException;

import de.w_hs.dbi.pa9.database.ConnectionInformation;
import de.w_hs.dbi.pa9.database.DatabaseConnection;
import de.w_hs.dbi.pa9.function.ProgramStage;

/**
 *  Zuweisungen an Threads.
 *  
 *  @author Mario Kellner
	@author Markus Hausmann
 *  @author Jonas Stadtler
 *
 */
public class ClientThread extends Thread {
	private DatabaseConnection threadCon;
	static boolean startTrans = false;
	private int threadID;
	private static int nextThreadID = 1;
	
	private synchronized static int getNextId() {
		return nextThreadID++;
	}


	
	public ClientThread(ConnectionInformation infos) throws Exception {
		threadCon = new DatabaseConnection(infos);
		threadID = getNextId();
	}
	
	static synchronized void setStartTrans () {
		startTrans = true;
	}

	synchronized boolean getStartTrans () {
		return startTrans;
	}

	public void run() {
		try {
			threadCon.connect();
			
			if(threadCon.databaseLink.isValid(0))
			{
				System.out.println("[Thread " + threadID +"]Thread erfolgreich zur Datenbank verbunden!");
			}
			else
			{
				throw new SQLException("Keine Verbindung zur Datenbank möglich");
			}
			
			System.out.println("[Thread " + threadID +"]Auf Startsignal warten ...");
			
			while(!getStartTrans()) {
				yield();
			}
			
			System.out.println("[Thread " + threadID +"]Startsignal erhalten!");
			
			
			ProgramStage program = new ProgramStage(threadCon);
			

			System.out.println("[Thread " + threadID +"]Führe Funktion attackStage aus!");
			program.attackStage();
			
			System.out.println("[Thread " + threadID +"]Führe Funktion benchStage aus!");
			program.benchStage();
			
			System.out.println("[Thread " + threadID +"]Führe Funktion boomOutStage aus!");
			program.boomOutStage();
		}
		catch (SQLException sqle)
		{
			System.out.println("Datenbank Fehler: " + sqle.getMessage());
			sqle.printStackTrace();
			
			Main.abortProgramm();
		}
		catch (Exception e)
		{
			System.out.println("Ein Fehler ist aufgetreten: " + e.getMessage());
			e.printStackTrace();

			Main.abortProgramm();
		}
			
	}
}
\end{lstlisting}


\begin{lstlisting}[caption={ProgramStage}, label={lst:pr3}]
package de.w_hs.dbi.pa9.function;

import java.sql.SQLException;

import de.w_hs.dbi.pa9.Main;
import de.w_hs.dbi.pa9.database.DatabaseConnection;

/**
 * 
 * Diese Klasse enthält alle die drei Schritte des Programmablaufs. Zudem werden hier die Abfrage
 * Werte per Zufall definiert und die Gewichtung erfolgt im geforderten Verhältnis.
 *
 * @author Mario Kellner
 * @author Markus Hausmann
 * @author Jonas Stadtler
 * 
 *
 */
public class ProgramStage 
{
	private int TransactionNumber;
	private int AccountsNumber;
	private int BranchesNumber;
	private int TellersNumber;
	private int deltaNumber;
	private DatabaseConnection connection;
	private TXHandler txh;
	
	public ProgramStage(DatabaseConnection con) throws SQLException
	{
		if(con == null)
		{
			throw new NullPointerException("Connection darf nicht NULL sein!");
		}
		
		this.connection = con;
		this.txh = new TXHandler(con);
	}
	
	/**
	 * Entscheidet per gewichteten Zufall darüber, welche Abfrage an die Datenbank geschickt wird..
	 * 
	 * @return Gibt die Zahl 1, 2 oder 3 zurück. Dadurch wird im folgenden über die Abfrage entschieden.
	 */
	public int choose()
	{
		int number=(int)(Math.random()*100);
		if(number <= 35)
		{
			return 1;
		}
		else if( number > 35 && number <= 85)
		{
			return 2;
		}
		return 3;
	}
	
	/**
	 * Entscheidet über die jeweiligen Zahlen, welche mit der Abfrage zur Datenbank geschickt werden.
	 * @param name Bezeichnung der anzusprechenden Tabelle.
	 * @param n Größe der Datenbank in TPS.
	 * @return Zufällige, aber passende Übergabewerte, welche zur Abfrage an die Datenbank 
	 * verwendet werden.
	 */
	private int chooseNumber(String name, int n)
	{
		double number=Math.random();
		switch(name)
		{
			case "accounts":
				number*=10000*n;
				break;
			case "branches":
				number*=n;
				break;
			case "tellers":
				number*=10*n;
				break;
			case "delta":
				number*=10000;
				break;
		}
		return (int)(number+1);
	}
	
	public void thinkTime() throws InterruptedException {
		long startTime = System.currentTimeMillis();
		
		this.generateNumbers();
		
		while(System.currentTimeMillis()-startTime < 50) {
			Thread.sleep(1);
		}
	}
	
	public void generateNumbers() {
		this.TransactionNumber = choose();
		this.AccountsNumber = chooseNumber("accounts", 100);
		this.TellersNumber = chooseNumber("tellers", 100);
		this.BranchesNumber = chooseNumber("branches", 100);
		this.deltaNumber = chooseNumber("delta", 100);
	}
	
	
	/**
	 * Entscheidet über die auszuführende Abfrage. 
	 * Die zufälligen Werte werden von der Funktion generateNumbers berechnet.
	 * 
	 * @throws SQLException Fehler, welcher bei der Abfrage an die Datenbank auftreten kann.
	 */
	private void doTX() throws SQLException
	{
		switch(this.TransactionNumber)
		{
			case 1:
				txh.kontostandTX(this.AccountsNumber);
				break;
			case 2:
				txh.einzahlungTX(
						this.AccountsNumber,
						this.TellersNumber,
						this.BranchesNumber,
						this.deltaNumber
				);
				break;
			case 3:
				txh.analyseTX(this.deltaNumber);
				break;
		}
	}
	
	/**
	 * Ausführen des ersten Programmteils.
	 * 
	 * @throws SQLException Möglicher Fehler bei der DB-Abfrage.
	 * @throws InterruptedException Möglicher Fehler bei der Wartezeit.
	 */
	public void attackStage() throws SQLException, InterruptedException
	{	
		long start = System.currentTimeMillis();
		
		// generiere Initialnummern
		this.generateNumbers();
		
		while(start + 240000 >= System.currentTimeMillis())
		{
			doTX();
			this.thinkTime();
		}
	}
	
	/**
	 * Ausführen des zweiten Programmteils.
	 * 
	 * @throws SQLException Möglicher Fehler bei der DB-Abfrage.
	 * @throws InterruptedException Möglicher Fehler bei der Wartezeit.
	 */
	public void benchStage() throws SQLException, InterruptedException
	{
		int txCounter = 0;
		long start = System.currentTimeMillis();
		
		// generiere Initialnummern
		this.generateNumbers();
		
		while(start + 300000 >= System.currentTimeMillis())
		{
			doTX();
			txCounter++;
			this.thinkTime();	
		}
		
		Main.setTxCountSum(Main.getTxCountSum() + txCounter);
		
		System.out.println("Anzahl der Transaktionen: " + txCounter);
		System.out.println("Anzahl der Transaktionen pro Sekunde: " + (double)((double)txCounter/(double)300));
		
	}
	
	/**
	 * Ausführen des dritten Programmteils.
	 * 
	 * @throws SQLException  Möglicher Fehler bei der DB-Abfrage.
	 * @throws InterruptedException Möglicher Fehler bei der Wartezeit.
	 */
	public void boomOutStage() throws SQLException, InterruptedException
	{
		long start = System.currentTimeMillis();
		
		// generiere Initialnummern
		this.generateNumbers();
		
		while(start + 60000 >= System.currentTimeMillis())
		{
			doTX();
			this.thinkTime();
		}
	}
}

\end{lstlisting}


\begin{lstlisting}[caption={TXHandler}, label={lst:pr4}]
package de.w_hs.dbi.pa9.function;

import java.sql.PreparedStatement;
import java.sql.ResultSet;
import java.sql.SQLException;
import java.sql.Statement;

import org.postgresql.util.PSQLException;

import de.w_hs.dbi.pa9.database.DatabaseConnection;

/**
 * In dieser Klasse sind die Statements hinterlegt, welche an die Datenbank geschickt werden.
 * 
 * @author Mario Kellner
 * @author Markus Hausmann
 * @author Jonas Stadtler
 *
 */
public class TXHandler 
{
	private DatabaseConnection connection;
	private Statement statement;
	
	public TXHandler(DatabaseConnection con) throws SQLException
	{
		this.connection=con;
		statement = connection.databaseLink.createStatement();
	}
	
	/**
	 * Abfrage eines Kontostandes.
	 * 
	 * @param accID Account ID 
	 * @return Kontostand der gesuchten Account ID
	 * @throws SQLException Fehler bei der Datenbank Abfrage, mit spezifizierter Fehlermeldung.
	 */
	public double kontostandTX(int accID) throws SQLException
	{
		Statement statement = connection.databaseLink.createStatement();
		
		ResultSet rs = statement.executeQuery("SELECT balance FROM accounts WHERE accid = " + accID);
		
		if(rs.next() == false)
		{
			throw new SQLException("Kontonummer nicht vorhanden!");
		}
		
		return rs.getDouble(1);
	}
	
	/**
	 * Ermittlung eiens neuen Kontostandes.
	 * 
	 * @return Neuer errechneter Kontostand.
	 * @throws SQLException Fehler bei der Datenbank Abfrage, mit spezifizierter Fehlermeldung.
	 */
	public double einzahlungTX(int accID, int tellerID, int branchID, double delta) throws SQLException
	{
		ResultSet rs = null;
		
		try {
			connection.databaseLink.setAutoCommit(false);
			
			statement.executeUpdate("UPDATE branches SET balance=balance+"+delta+" WHERE branchid="+branchID);
			statement.executeUpdate("UPDATE tellers SET balance=balance+"+delta+" WHERE tellerid="+tellerID);
			statement.executeUpdate("UPDATE accounts SET balance=balance+"+delta+" WHERE accid="+accID);
			statement.executeUpdate("INSERT INTO history (accid, tellerid, delta, branchid, accbalance, cmmnt)"
					+ "VALUES("+accID+", "+tellerID+", "+delta+", "+branchID+", (SELECT balance FROM accounts WHERE accid="+accID+"), ' ')");
			
			rs = statement.executeQuery("SELECT balance FROM accounts WHERE accid = " + accID);
			
			connection.databaseLink.commit();
			
			if(rs.next() == false)
			{
				throw new SQLException("Kontonummer nicht vorhanden");
			}
			
		}
		catch (PSQLException e) {
			System.out.println(e.getMessage());
			connection.databaseLink.rollback();
			
		}
		finally {

			connection.databaseLink.setAutoCommit(true);
		}
		
		return rs.getDouble(1);
	}
	
	/**
	 * Anzahl der Einzahlungen mit dem Betrag delta sollen ermittelt werden.
	 * @param delta Einzahlungsbetrag
	 * @return Anzahl der Einzahlungen mit dem Betrag delta wird zurückgegeben.
	 * @throws SQLException Fehler bei der Datenbank Abfrage, mit spezifizierter Fehlermeldung.
	 */
	public int analyseTX(double delta) throws SQLException
	{
		Statement statement = connection.databaseLink.createStatement();
		
		ResultSet rs = statement.executeQuery("SELECT COUNT(*) FROM history WHERE delta=" + delta);
		
		if(rs.next() == false)
		{
			throw new SQLException("Kein Eintrag mit dem Wert " + delta + " vorhanden!");
		}
		
		return rs.getInt(1);
	}
}
\end{lstlisting}


\begin{lstlisting}[caption={DatabaseConnection}, label={lst:pr5}]
package de.w_hs.dbi.pa9.database;

import java.sql.Connection;
import java.sql.DriverManager;
import java.sql.SQLException;
import java.sql.Statement;

/**
 * Unsere Datebankklasse ermöglicht die Kommunikation
 * mit der Datenbank. Es ist ein Wrapper für das
 * Connection-Objekt der JDBC Connection
 * 
 * @author Mario Kellner
 * @author Markus Hausmann
 * @author Jonas Stadtler
 */
public class DatabaseConnection 
{
	/**
	 * Das eigentliche Connection Objekt
	 */
	public Connection databaseLink;
	
	/**
	 *  Unser Verbindungs-Model
	 */
	private ConnectionInformation ci;
	
	/**
	 * Der Konstruktur der Klasse.  Überprüft ob das Objekt im C
	 * 
	 * @param cInfo Das Connection Objekt
	 * @throws Exception Wird geworfen, wenn das Objekt was übergeben wird null ist.
	 */
	public DatabaseConnection(ConnectionInformation cInfo) throws Exception {
		if(cInfo == null) {
			throw new Exception("ConnectionInfo Object cannot be null");
		}
		
		ci = cInfo;
	}
	public void clearHistory() throws SQLException
	{
		Statement statement= databaseLink.createStatement();

		statement.executeUpdate("TRUNCATE TABLE history");
	}
	/**
	 * 
	 * List die Informatioen aus dem Objekt ci (ConnectionInformation) und versucht eine Verbindung 
	 * zur Datenbank auf zu bauen.
	 * 
	 * @throws SQLException Wird geworfen, wenn der DriverManager keine Verbndung
	 zur Datenbank aufbauen kann */
	public void connect() throws SQLException
	{
		databaseLink = DriverManager.getConnection(
				"jdbc:postgresql://" + ci.getHost() +"/" + ci.getDatabase() + "?useCompression=true",
				ci.getUser(), 
				ci.getPassword()
		);
	}
}
\end{lstlisting}

\begin{lstlisting}[caption={ConnectionInformation}, label={lst:pr6}]
package de.w_hs.dbi.pa9.database;

/**
 * Dieses Model beinhaltet alle nötigen Informationen 
 * um sich zu einer Postgre Datenbank zu verbinden
 * 
 * @author Mario Kellner
 * @author Markus Hausmann
 * @author Jonas Stadtler
 *
 */
public class ConnectionInformation {
	
	// Eigenschaften
	public String host = null;
	public String database = null;
	public String user = null;
	public String password = null;
	
	
	// Getter & Setter
	public String getHost() {
		return host;
	}
	public void setHost(String host) {
		this.host = host;
	}
	public String getDatabase() {
		return database;
	}
	public void setDatabase(String database) {
		this.database = database;
	}
	public String getUser() {
		return user;
	}
	public void setUser(String user) {
		this.user = user;
	}
	public String getPassword() {
		return password;
	}
	public void setPassword(String password) {
		this.password = password;
	}
}
\end{lstlisting}
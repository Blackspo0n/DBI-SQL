\usepackage[ngerman]{babel}
\usepackage[T1]{fontenc}
\usepackage{textcomp} % Euro-Zeichen etc.
\usepackage{lmodern} % bessere Fonts
\usepackage{relsize} % Schriftgröße relativ festlegen
\renewcommand{\familydefault}{\sfdefault}

\PassOptionsToPackage{table}{xcolor}

\usepackage{tabularx}
\usepackage{longtable}
\usepackage{array}
\usepackage{ragged2e}
\usepackage{lscape}
\newcolumntype{w}[1]{>{\raggedleft\hspace{0pt}}p{#1}} % Spaltendefinition rechtsbündig mit definierter Breite
\usepackage[dvips,final]{graphicx} % Einbinden v on JPG-Grafiken ermöglichen
\usepackage{graphics} % keepaspectratio
\usepackage{floatflt} % zum Umfließen von Bildern
\graphicspath{{Bilder/}} % hier liegen die Bilder des Dokuments
\usepackage{amsmath,amsfonts} % Befehle aus AMSTeX für mathematische Symbole
\usepackage{enumitem} % anpassbare Enumerates/Itemizes
\usepackage{xspace} % sorgt dafür, dass Leerzeichen hinter parameterlosen Makros nicht als Makroendezeichen interpretiert werden

\usepackage{makeidx} % für Index-Ausgabe mit \printindex
\usepackage{acronym} % es werden nur benutzte Definitionen aufgelistet
\usepackage[table]{xcolor}
\usepackage{hhline}
\usepackage{setspace}
\usepackage{geometry}
\usepackage{eso-pic}

% Symbolverzeichnis
\usepackage[intoc]{nomencl}
\let\abbrev\nomenclature
\renewcommand{\nomname}{Abkürzungsverzeichnis}
\setlength{\nomlabelwidth}{.25\hsize}
\renewcommand{\nomlabel}[1]{#1 \dotfill}
\setlength{\nomitemsep}{-\parsep}

\usepackage{url} % URL verlinken, lange URLs umbrechen etc.
\usepackage{chngcntr} % fortlaufendes Durchnummerieren der Fußnoten
\usepackage{ifthen} % bei der Definition eigener Befehle benötigt
\usepackage{todonotes} % definiert u.a. die Befehle \todo und \listoftodos
\usepackage[square]{natbib} % wichtig für korrekte Zitierweise

\usepackage{pdfpages}
\usepackage{tikz}
\usepackage{arrayjob}
\usepackage{xmpincl}
\includexmp{copyright}

\pdfminorversion=5 % erlaubt das Einfügen von pdf-Dateien bis Version 1.7, ohne eine Fehlermeldung zu werfen (keine Garantie für fehlerfreies Einbetten!)
\usepackage[
    bookmarks,
    bookmarksnumbered,    
    bookmarksopen=true,
    bookmarksopenlevel=1,
    colorlinks=true, % diese Farbdefinitionen zeichnen Links im PDF farblich aus
    linkcolor=WHSGreen, % einfache interne Verknüpfungen
    anchorcolor=WHSGreen,% Ankertext
    citecolor=WHSGreen, % Verweise auf Literaturverzeichniseinträge im Text
    filecolor=WHSGreen, % Verknüpfungen, die lokale Dateien öffnen
    menucolor=WHSGreen, % Acrobat-Menüpunkte
    urlcolor=WHSGreen, % diese Farbdefinitionen sollten für den Druck verwendet werden (alles% schwarz)
    pdftex,
    plainpages=false, % zur korrekten Erstellung der Bookmarks
    pdfpagelabels=true, % zur korrekten Erstellung der Bookmarks
    hypertexnames=false, % zur korrekten Erstellung der Bookmarks
    linktocpage % Seitenzahlen anstatt Text im Inhaltsverzeichnis verlinken
]{hyperref}


\hypersetup{
    pdftitle={\titel},
    pdfauthor={\autor},
    pdftrapped={true},
    pdfcreator={Texlipse},
    pdfsubject={\untertitel},
    pdfkeywords={\keywords},
    pdfproducer={pdftex},
}

\usepackage{listings}
\usepackage{xcolor} 
\usepackage{subfigure} 
\definecolor{hellgelb}{rgb}{1,1,0.9}
\definecolor{colKeys}{rgb}{0,0,1}
\definecolor{colIdentifier}{rgb}{0,0,0}
\definecolor{colComments}{rgb}{0,0.5,0}
\definecolor{colString}{rgb}{1,0,0}

\lstset{
    float=hbp,
	basicstyle=\footnotesize,
    identifierstyle=\color{colIdentifier},
    keywordstyle=\color{colKeys},
    stringstyle=\color{colString},
    commentstyle=\color{colComments},
    backgroundcolor=\color{hellgelb},
    columns=flexible,
    tabsize=2,
    frame=single,
    extendedchars=true,
    showspaces=false,
    showstringspaces=false,
    numbers=left,
    numberstyle=\tiny,
    breaklines=true,
    breakautoindent=true,
	captionpos=b,
}
\lstdefinelanguage{cs}{
	sensitive=false,
	morecomment=[l]{//},
	morecomment=[s]{/*}{*/},
	morestring=[b]",
	morekeywords={
		abstract,event,new,struct,as,explicit,null,switch
		base,extern,object,this,bool,false,operator,throw,
		break,finally,out,true,byte,fixed,override,try,
		case,float,params,typeof,catch,for,private,uint,
		char,foreach,protected,ulong,checked,goto,public,unchecked,
		class,if,readonly,unsafe,const,implicit,ref,ushort,
		continue,in,return,using,decimal,int,sbyte,virtual,
		default,interface,sealed,volatile,delegate,internal,short,void,
		do,is,sizeof,while,double,lock,stackalloc,
		else,long,static,enum,namespace,string},
}
\lstdefinelanguage{natural}{
	sensitive=false,
	morecomment=[l]{/*},
	morestring=[b]",
	morestring=[b]',
	alsodigit={-,*},
	morekeywords={
		DEFINE,DATA,LOCAL,END-DEFINE,WRITE,CALLNAT,PARAMETER,USING,
		IF,NOT,END-IF,ON,*ERROR-NR,ERROR,END-ERROR,ESCAPE,ROUTINE,
		PERFORM,SUBROUTINE,END-SUBROUTINE,CONST,END-FOR,END,FOR,RESIZE,
		ARRAY,TO,BY,VALUE,RESET,COMPRESS,INTO,EQ},
}
\lstdefinelanguage{php}{
	sensitive=false,
	morecomment=[l]{/*},
	morestring=[b]",
	morestring=[b]',
	alsodigit={-,*},
	morekeywords={
		abstract,and,array,as,break,case,catch,cfunction,class,clone,const,
		continue,declare,default,do,else,elseif,enddeclare,endfor,endforeach,
		endif,endswitch,endwhile,extends,final,for,foreach,function,global,
		goto,if,implements,interface,instanceof,namespace,new,old_function,or,
		private,protected,public,static,switch,throw,try,use,var,while,xor
		die,echo,empty,exit,eval,include,include_once,isset,list,require,
		require_once,return,print,unset},
}

\definecolor{editorGray}{rgb}{0.95, 0.95, 0.95}

\pdftrailerid{}
\pdfsuppressptexinfo15

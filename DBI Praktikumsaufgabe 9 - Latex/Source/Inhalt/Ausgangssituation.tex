\section{Ausgangssituation}
Am Datenbankmanagement-System (DBMS) sind einige Einstellungen über die postresql.conf änderbar. In 
dieser Datei sind die Einstellungen des DBMS gespeichert und können vom Benutzer editiert werden. 

Zunächst wird von uns die Anzahl der Verbindungen, welche gleichzeitig zur Datenbank bestehen 
dürfen, (max_connections) von voreingestellten 100 auf die von uns benötigten 18 heruntergesetzt. 
Dies soll einen Performancegewinn bringen, da zu viele erlaubte Verbindungen unnötig Ressourcen 
ziehen, zudem haben wir eine ideale Ausgangssituation, da auf 8 virtuellen Kernen lediglich 18 
Verbindungen existieren. Dies erfüllt die Idealbedingung von zwei bis drei erlaubten Verbindungen 
je Kern.

Wir haben uns im Weiteren mit der Pufferspeicherung in der .conf Datei befasst, welche 
shared_buffers heißt. Dazu wird dem Pufferspeicher ¼ des Ram zugewiesen. Der restliche Ram wird 
für den effective_cache_size verwendet, welcher der Planung und Ausführung von Querys dient.  

Zudem haben wir dem work_mem einen höheren Wert zugewiesen, eine Optimierung ist hierbei nur 
möglich, wenn man die Anzahl der maximalen Verbindungen kennt. Wir haben uns für den Wert 
RAM/(max_connections * 16) entschieden, da dieser sich aufgrund von Internetrecherche und nach 
einigen Tests als effektiv erwiesen hat. Work_mem ist für die Verknüpfung von Tabellen oder 
Umsetzung bestimmter Klauseln wichtig und weist zu, wie viel Speicher je Sortier-und Hashoperation 
verwendet werden kann.


\clearpage
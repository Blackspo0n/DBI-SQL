\section{Ausgangssituation}
Die Aufgabenstellung sieht vor, dass ein Programm entwickelt wird, mit dem es
möglich sein soll Lasten gegen unser gewähltes DBMS zu setzen. 

Zur Optimierung des geforderten Programms haben wir einige Lösungsansätze
verfolgt, welche im Folgenden erläutert werden. Zunächst werden unsere
grundsätzlichen Überlegungen eines Optimierungsansatzes dargestellt. Dieser
Ansatz orientiert sich an den vorgegeben Rahmenbedingungen und bildet die
grundlegende Überlegung bei der Entwicklung unseres Programms.

Aufgrund der Vorgaben ist die Möglichkeit einer Verbesserung an lediglich drei
Stellen gegeben. So könnte man das Absenden der Statements bzw. deren Inhalt,
die zufällige Auswahl der Daten innerhalb der Statements oder die internen
Einstellungen der Datenbank optimieren, um einen höheren Durchsatz zu erzielen.

Der restliche Teil des Programms ist aufgrund der Vorgabe nicht änderbar oder
wird vor dem Ausführen der Messungen durchgeführt. Die Vorgaben verlangsamen
das Programm, denn nach jeder der auszuführenden Operationen wird eine Pause
von \textbf{50 Millisekunden} eingelegt, in welcher das Programm abwartet und
keine Transaktionen starten darf.Zudem muss über alle Werte und Operationen per
Zufall entschieden werden. Aufgrund der Vorgaben konvergiert die maximale
Abfrage je Sekunde pro Thread gegen \textbf{20}.

Bei der Anfrageoptimierung ist es wichtig, möglichst wenig Daten abzufragen, um
unnötigen Datentransfer zu verhindern. Zudem sollte darauf geachtet werden
möglichst wenige Abfragen durchzuführen, dies kann zum Beispiel durch das
Verwenden von Subselects erreicht werden. Um die Abfrage noch weiter zu
verbessern ist es außerdem wichtig die Datenbanksprache SQL gut zu kennen und
möglichst viele Operationen von dieser durchführen zu lassen. Dies spart nicht
nur unnötigen Code ein, sondern auch Datentransfer.

Eine weitere Möglichkeit zur Optimierung des Programms ist es, die Daten welche
zufällig für die Abfragen ausgewählt werden, während der 50 Millisekunden
langen Pause nach jeder Transaktion bestimmen zu lassen. 

Die dritte Möglichkeit sieht vor die Einstellungen der Datenbank zu optimieren.
Diese Optimierungen wirken sich auf die Ressourcennutzung, sowie die
Nachverfolgbarkeit der Datenbankeinträge bzw. Abfragen aus. Dabei sind die
technischen Voraussetzungen, welche der verwendete Computer bietet von großer
Bedeutung. Besonders wichtig ist dabei die Art der Festplatte, so ist eine
\textbf{SSD-Festplatte} in der Lage deutlich schneller Daten zu schreiben oder
zu lesen als eine \textbf {HDD-Festplatte}. Aber nicht nur die Festplatte ist von
Bedeutung, denn ein Puffer, welcher beim Schreiben besonders wichtig ist, benötigt
Arbeitsspeicher. Des Weiteren, ist ein starker Prozessor von Vorteil, da dieser
ein schnelleres Ausführen des Programms ermöglicht.

\clearpage
\section{Ausgangssituation}

Aufgrund der Aufgabenstellung ist die Möglichkeit einer Verbesserung an
lediglich drei Stellen gegeben. So könnte man das Absenden der Statements bzw.
deren Inhalt, die zufällige Auswahl der Daten innerhalb der Statements oder die
internen Einstellungen der Datenbank optimieren, um einen höheren Durchsatz zu
erzielen.

Der restliche Ablauf des Programms ist aufgrund der vorgegebene \gqq{ThinkTime}
nicht änderbar oder wird vor dem Ausführen der Messungen durchgeführt. Die Vorgaben machen das
Programm zudem beabsichtigt langsamer als es sein müsste. Denn nach jeder
Abfrage bzw. Update wird eine Pause von 50ms eingelegt, in welcher das Programm
abwartet und keine Transaktionen starten darf. Zudem muss über alle Werte und
Operationen per Zufall entschieden werden. Aufgrund der Vorgaben konvergiert
die maximale Abfrage je Sekunde, bei fünf nebenläufigen Load-Driver-Programmen,
gegen 20.

Bei der Anfrageoptimierung ist es wichtig, möglichst wenig Daten abzufragen, um
unnötigen Datentransfer zu verhindern. Zudem sollte darauf geachtet werden
möglichst wenige Abfragen durchzuführen, dies kann zum Beispiel durch das
Verwenden von Subselects erreicht werden. Um die Abfrage noch weiter zu
verbessern ist es außerdem wichtig die Datenbanksprache SQL gut zu kennen und
möglichst viele Operationen von dieser bearbeiten zu lassen. Dies spart nicht
nur unnötigen Code ein, sondern spart auch Abfragen und Datentransfer.

Eine weitere Möglichkeit zur Optimierung des Programms ist es, die Daten welche
zufällig für die Abfragen ausgewählt werden nicht über die Funktion
MATH.Random() zu bestimmen. Das Gleiche gilt auch für die verteilte Auswahl der
auszuführenden Operationen. Ein Lösungsansatz wäre es, die einzutragenden Daten
vor dem Start des Benchmarkings in ein Array oder .txt Datei einzutragen und
diese dann lediglich an den entsprechenden Stellen einzutragen. Für die Auswahl
der Operation gilt das Gleiche Prinzip. So kann man vor dem Start des
Benchmarkings ein Array mit Zahlen von eins bis 100 Füllen. Bei der Abfrage in
der Schleife können dann die Zahlen eins bis 35 für die Auswahl des
Kontostandes verwendet werden, 36 bis 85 für die Auswahl Einzahlung und 86 bis
100 für die Auswahl des Kontostandes stehen. Dabei ist die Voraussetzung eines
zufällig Ausgewählten Vorgangs in der zehnminütigen Schleife des
Load-Driver-Programms erfüllt. Denn das Programm verwendet zufällige Werte,
welche die geforderten Wahrscheinlichkeiten erfüllen. Zudem erfolgt die
Auswertung, der Werte und die Entscheidung welche Funktion gestartet wird
innerhalb der Schleife.

Die dritte Möglichkeit ist es, die Einstellungen der Datenbank zu Optimieren.
Diese Optimierung wirkt sich auf die Ressourcennutzung, sowie die
Nachverfolgbarkeit der Datenbankeinträge bzw. Abfragen aus. Dabei sind die
technischen Voraussetzungen, welche der verwendete Computer bietet von großer
Bedeutung. Besonders wichtig ist dabei die Art der Festplatte, so ist eine
SSD-Festplatte in der Lage deutlich schneller Daten zu schreiben oder zu lesen,
als eine HDD-Festplatte. Aber nicht nur die Festplatte ist von Bedeutung, denn
ein Puffer, welcher beim Schreiben besonders wichtig ist, benötigt
Arbeitsspeicher. Des Weiteren, ist ein s tarker Prozessor von Vorteil, da dieser
ein schnelleres Ausführen des Programms ermöglicht.

\clearpage
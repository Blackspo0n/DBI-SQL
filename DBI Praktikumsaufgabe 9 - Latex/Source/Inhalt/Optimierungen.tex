\section{Optimierungen im Programm}
Das erstellte Programm war bereits in der ersten Version effizient. Denn die
benötigten SQL-Befehle waren von Anfang an optimiert. Bei der Erstellung der
SQL-Befehle wurde darauf geachtet möglichst wenige Befehle mit möglichst wenig
abgefragten Variablen zu erstellen.
\lstset{language=Java, backgroundcolor=\color{editorGray},
  basicstyle={\linespread{0.82}\footnotesize\ttfamily},   
  frame=b, xleftmargin={0.75cm},literate=
    {Ö}{{\"O}}1
  {Ä}{{\"A}}1
  {Ü}{{\"U}}1
  {ß}{{\ss}}2
  {ü}{{\"u}}1
  {ä}{{\"a}}1
  {ö}{{\"o}}1,
    numberstyle=\tiny\noncopynumber,
      }
\begin{lstlisting}[caption={Statement der Funktion kontostandTX in
TXHandler.java}] 
   ResultSet rs = statement.executeQuery("SELECT balance FROM accounts WHERE accid = " + accID);
\end{lstlisting}

Durch das Verwenden des Operators \textbf{count (*)} sparen wir uns eine
Schleife im Programm, welche die Anzahl der Elemente, mit der geforderten Bedingung,
ermittelt. Dabei wird auch Zeit gespart, da sich lediglich ein Element im
ResultSet befindet und auch nur eins aufgerufen werden muss. 
\lstset{language=Java, backgroundcolor=\color{editorGray},
  basicstyle={\linespread{0.82}\footnotesize\ttfamily},   
  frame=b, xleftmargin={0.75cm},literate=
    {Ö}{{\"O}}1
  {Ä}{{\"A}}1
  {Ü}{{\"U}}1
  {ß}{{\ss}}2
  {ü}{{\"u}}1
  {ä}{{\"a}}1
  {ö}{{\"o}}1,
    numberstyle=\tiny\noncopynumber,
      }
\begin{lstlisting}[caption={Statement der Funktion analyseTX in
TXHandler.java}] 
  ResultSet rs = statement.executeQuery("SELECT COUNT(*) FROM history WHERE delta=" + delta);
\end{lstlisting}
 
Eine weitere Optimierung ist die Verwendung eines \textbf{Subselects} beim
Insert-Befehl.
Dieser ermöglicht es auf einen Select-Befehl verzichten zu können.
\lstset{language=Java, backgroundcolor=\color{editorGray},
  basicstyle={\linespread{0.82}\footnotesize\ttfamily},   
  frame=b, xleftmargin={0.75cm},literate=
    {Ö}{{\"O}}1
  {Ä}{{\"A}}1
  {Ü}{{\"U}}1
  {ß}{{\ss}}2
  {ü}{{\"u}}1
  {ä}{{\"a}}1
  {ö}{{\"o}}1,
    numberstyle=\tiny\noncopynumber,
      }
\begin{lstlisting}[caption={Statement der Funktion einzahlungTX in
TXHandler.java}] 
  statement.executeUpdate("UPDATE branches SET balance=balance+"+delta+" WHERE branchid="+branchID);
  statement.executeUpdate("UPDATE tellers SET balance=balance+"+delta+" WHERE tellerid="+tellerID);
  statement.executeUpdate("UPDATE accounts SET balance=balance+"+delta+" WHERE accid="+accID);
  statement.executeUpdate("INSERT INTO history (accid, tellerid, delta, branchid, accbalance, cmmnt)"
       + "VALUES("+accID+", "+tellerID+", "+delta+", "+branchID+", (SELECT balance FROM accounts WHERE accid="+accID+"), ' ')");
            
            rs = statement.executeQuery("SELECT balance FROM accounts WHERE accid = " + accID);
\end{lstlisting}

Die Verwendung von einfachen \textbf{excecuteUpdate-Befehlen} macht in diesem
Fall mehr Sinn als das Verwenden von PreparedStatements. Denn wir müssten bei der
Verwendung von PreparedStatements für jede einzelne Tabelle, welche wir
anspreche wollen, ein neues PreparedStatements erstellen um ein Update
durchzuführen. Dadurch würden wir keine höhere Effizienz erreichen.

Die einzig effiziente Änderung die wir durchgeführt haben ist, dass wir die
Zufallswerte innerhalb der 50ms Pause durchgeführt wird und wir dadurch je
Durchgang einige tausendstel Sekunden sparen.

\clearpage
\section{Optimierungen im Programm}
Das erstellte Programm war bereits in der ersten Version effizient. Denn die
benötigten SQL-Befehle waren von Anfang an optimiert. Bei der Erstellung der
SQL-Befehle wurde darauf geachtet möglichst wenige Befehle mit möglichst wenig
abgefragten Variablen  zu erstellen.

(Quelltext, wo die SQL Befehle gezeigt werden)

Durch das Verwenden des Operators count (*) sparen wir uns eine Schleife im
Programm, welche die Anzahl der Elemente, mit der geforderten Bedingung,
ermittelt. Dabei wird auch Zeit gespart, da sich lediglich ein Element im
ResultSet befindet und auch nur eins aufgerufen werden muss. Eine weitere
Optimierung ist die Verwendung eines Subselects beim Insert-Befehl. Dieser
ermöglicht es auf einen Select-Befehl verzichten zu können.

Die Verwendung von einfachen excecuteUpdate-Befehlen macht in diesem Fall mehr
Sinn als das Verwenden von PreparedStatements. Denn wir müssten bei der
Verwendung von PreparedStatements für jede einzelne Tabelle, welche wir
anspreche wollen, ein neues PreparedStatements erstellen um ein Update
durchzuführen. Dadurch würden wir keine höhere Effizienz erreichen.

Die einzig effiziente Änderung die wir durchgeführt haben ist, dass wir die
Zufallswerte innerhalb der 50ms Pause durchgeführt wird und wir dadurch je
Durchgang einige tausendstel Sekunden sparen.

\clearpage
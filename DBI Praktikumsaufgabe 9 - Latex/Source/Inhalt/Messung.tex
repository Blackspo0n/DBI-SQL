\section{Ausführung der Messung}

Zum Ausführen von 5 remote Load Driver werden zunächst 5 Threads erzeugt, welche die Nebenläufigkeit 
ermöglichen. Für jeden Thread wird eine Verbindung zur Datenbank erstellt und zugewiesen, des 
Weiteren wird noch eine Verbindung für den Main-Thread erzeugt, welche lediglich zum Löschen der 
Tabelle history verwendet wird. Nach dem leeren der Tabelle beginnen die Threads gleichzeitig mit 
der Einschwingphase und gehen nach dieser in die Messphase über, um nach der folgenden 
Ausschwingphase das Programm zu beenden.

Mit dem Start der Einschwingphase beginnen die Threads mit der zufälligen Auswahl einer Operation 
und der damit verbundenen Transaktion, dies erfolgt unabhängig voneinander. Nach der Auswahl der 
Abfrage werden die Zufallsdaten für die Abfrage mit Hilfe der Funktion
\textbf{chooseNumber()} ermittelt, welche die Informationen über die benötigt
Variable übermittelt bekommt. Durch das Übergeben der Bezeichnung der benötigten Variablen und der Größe der Datenbank, soll gewährleistet sein, dass es 
zu keinen fehlerhaften Abfragen und einer dadurch fehlerhaften Messung kommt. Die Zahlen werden von 
der Funktion im zulässigen Bereich zufällig ausgewählt und sind unabhängig von den nebenläufigen 
Threads. Nach dem Ausführen der Anfrage wird eine Pause von 50ms eingelegt, während dieser Pause 
werden die Zufallsdaten für die nächste Abfrage ermittelt. In der Messphase wird vor der Pause eine 
Variable um einen erhöht, welche die Anzahl der Abfragen zählt und im weiteren Verlauf, mit Hilfe 
von synchronized, die Anzahl aller Abfragen aus den 5 Threads in der Messphase addiert. Diese 
Variable ist die Grundlage für die Rechnung der Anfragen je Sekunde.

Bei der Ausführung auf mehreren Computern werden die Threads zu einem vordefinierten Zeitpunkt, 
gleichzeitig auf allen Computern, gestartet. Die Ermittlung der Anfragen je Sekunde erfolgt nach 
Beendigung der Programme händisch, durch das Addieren der Ergebnisse der drei Anwendungen.


\clearpage
\section{Ausführung der Messung}

Zum Ausführen von 5 remote Load Driver wird zunächst für alle 5 erzeugten
Threads, welche die Nebenläufigkeit ermöglichen, eine Verbindung zur Datenbank
erzeugt. Diese Verbindung zur Datenbank wird zunächst für das leeren der
Tabelle history verwendet. Nach dem leeren der Tabelle beginnen die Threads
gleichzeitig mit der Einschwingphase und gehen nach dieser in die Messphase
über, um nach der folgenden Ausschwingphase das Programm zu beenden.

Mit dem Start der Einschwingphase beginnen die Threads mit der zufälligen
Auswahl einer Abfrage an die Datenbank, dies erfolgt unabhängig voneinander.
Nach der Auswahl der Abfrage werden die Zufallsdaten für die Abfrage mit Hilfe
der Funktion chooseNumber() ermittelt, welche die Informationen über die
benötigt Variable übermittelt bekommt. Durch das übergeben der Bezeichnung, der
benötigten Variablen, und der Größe der Datenbank, soll gewährleistet sein,
dass es zu keinen fehlerhaften Abfragen und einer dadurch fehlerhaften Messung
kommt. Die Zahlen werden von der Funktion im zulässigen Bereich zufällig
ausgewählt und sind unabhängig von den nebenläufigen Threads. Nach dem
Ausführen der Anfrage wird eine Pause von 50ms eingelegt. In der Messphase wird
vor der Pause eine Variable um einen erhöht, welche die Anzahl der Abfragen
zählt und im weiteren Verlauf, mit Hilfe von synchronized, die Anzahl aller
Abfragen aus den 5 Threads, in der Messphase, addiert. Diese Variable ist die
Grundlage für die Rechnung der Anfragen je Sekunde.

Bei der Ausführung auf mehreren Computern werden die Threads zu einem
vordefinierten Zeitpunkt,  gleichzeitig auf allen Computern, gestartet. Die
Ermittlung der Anfragen je Sekunde erfolgt nach Beendigung der Programme
händisch, durch das Addieren aller Anfragen und das Teilen dieser durch die 5
Minuten bzw. 300 Sekunden der Messphase.

\clearpage
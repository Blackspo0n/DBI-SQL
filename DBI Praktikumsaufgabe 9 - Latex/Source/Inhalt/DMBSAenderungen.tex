\section{Änderungen am Datenbank Management System}

Am Datenbankmanagement-System (DBMS) werden einige Einstellungen über die
postreSQL.conf ausgeführt. In dieser Datei sind die Einstellungen des DBMS
gespeichert und können vom Benutzer verändert werden.

Zunächst wird von uns die Anzahl der Verbindungen, welche gleichzeitig zur
Datenbank bestehen dürfen, (max\_connections) von voreingestellten 100 auf die
von uns benötigten 18 heruntergesetzt. Dies soll einen Performancegewinn
bringen, da zu viele Erlaubte Verbindungen unnötig Ressourcen ziehen, zudem
haben wir eine ideale Ausgangssituation, da auf 8 virtuellen Kernen lediglich
18 Verbindungen existieren. Dies erfüllt die Idealbedingung von zwei bis drei
erlaubten Verbindungen je Kern.

Wir haben uns im Weiteren mit der Pufferspeicherung befasst in der .conf Datei
shared\_buffers heißt. Dazu wird dem Pufferspeicher ein ¼ des Ram zugewiesen.
Der restliche Ram wird für den effective\_cache\_size verwendet, welcher der
Planung und Ausführung von Querys dient.

Zudem haben wir work\_mem einen höheren Wert zugewiesen, eine Optimierung ist
hierbei nur möglich, wenn man die Anzahl der benötigten Verbindungen kennt. Wir
haben uns für den Wert RAM/(max\_connections * 16) entschieden, da dieser sich
aufgrund von Internetrecherche und nach einigen Tests als effektiv erwiesen
hat. Work\_mem ist für die Verknüpfung von Tabellen oder Umsetzung bestimmter
Klauseln wichtig und wist zu, wie viel Speicher je Sortier-und Hashoperation
verwendet werden kann.

\clearpage
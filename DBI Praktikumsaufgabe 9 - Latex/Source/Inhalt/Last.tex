\section{Bewertung und Dokumentation der Lasten}

\subsection{Last 5}
Bei der ersten Last, also der Verwendung von 5 nebenläufigen
Load-Driver-Programmen auf einem PC, haben wir durch die Optimierungen eine
Verbesserung von 85 TPS auf insgesamt 95 erreicht. Nach der Verbesserung des
Clients, also der Ermittlung der Zufallsvariablen während der Pause von 50ms,
erreichten wir einen Wert von 93 TPS. Bei einigen anderen Messung erreichten
wir lediglich geringe Ergebnisse, da nebenher laufende Programme am
Test-Computer die Messergebnisse negativ beeinflusst haben.

Nach der Optimierung des Programms haben wir das DBMS optimiert und einen Wert
von 95 TPS erreicht. Dieser Wert ist stabil, weicht also bei weiteren Messungen
maximal um 0,5 TPS nach Unten oder Oben ab.


\subsection{Last 5x5}
Bei der zweiten Last, also der Verwendung von 5 nebenläufigen
Load-Driver-Programmen auf zwei Computern, welche Abfragen an die gleiche
Datenbank schicken, haben wir einen Bestwert von 185 TPS erzielt. Bei der
ersten Messung, also der Messung des nicht optimierten Programms, haben wir
einen Wert von 243 TPS erreicht, welchen wir durch die Optimierung des
Programms auf 181 TPS erhöhen konnten. Trotz Optimierung der Datenbank war es
nicht möglich, einen Wert über 185 TPS zu erreichen.

Dies ist für uns nicht nachvollziehbar, denn wir haben durch das Erhöhen des
Puffers und einer Anpassung der Einstellung auf die benötigte Funktionalität,
eine ideale Situation für die Abfrage und das Schreiben von Daten geschaffen.
Besonders bei einer größeren Menge von Anfragen haben wir mit einer deutlich
stärkeren Leistung des DBMS gerechnet, wenn man die Einstellungen optimiert.

Alle Ergebnisse, in jeder Phase der Entwicklung, waren stabil und es gab auch
keine Probleme mit dem Einfluss von nebenher laufenden Programmen. Jedoch ist
uns aufgefallen, dass immer ein Computer bessere Ergebnisse liefert als der
Andere. Daher vermuten wir, dass die Abfragen, je nachdem welcher Computer diese
abschickt, priorisiert oder zunächst zurückgestellt werden.

\clearpage

\subsection{Last 5x5x5}
Bei der dritten Last, also der Verwendung von 5 nebenläufigen
Load-Driver-Programmen auf drei Computern, welche Abfragen an die gleiche
Datenbank schicken, haben wir einen Bestwert von 267 TPS erzielt. Bei unserer
ersten Messung haben wir lediglich einen Wert 243 TPS erreicht, welchen wir
durch die Verbesserung des Programms um 9 TPS, auf 252 TPS, steigern konnten.
Diese Steigerung  ist im Verhältnis zu den beiden anderen Lasten eher gering.
Denn hier steigern wir die Leistungsfähigkeit des Programms und somit müsste
die Leistung um den dreifachen Wert steigen, als sie
dies bei Last eins, in der gleichen Situation, getan hat. Demnach müsste die
Anzahl der TPS statt um 9, um etwa 3 mal 8, also um etwa 24 steigen.

Die Steigerung von 252 TPS auf 267, bei der Optimierung des DBMS ist ebenfalls
geringer als erwartet. Denn unserer Überlegung nach müssten sich die
Auswirkungen der Optimierung besonders stark bei einer größeren Menge von
Abfragen zeigen. Dem ist aber folglich nicht so, dies könnte daran liegen, dass
das DBMS mit einer solchen Menge an Abfragen und den bereitgestellten
Ressourcen der virtuellen Maschine nicht auskommt und lediglich eine Erhöhung
der Ressourcen zu einem besseren Ergebnis führt.

Bei dieser Last war es uns leider nicht möglich so viele Messungen wie bei den
Anderen durchzuführen. Somit ist es nicht möglich Schwankungen der Ergebnisse
auszuschließen, jedoch sind uns keine besonders starken Abweichungen
aufgefallen.

\clearpage
\section{Implementierung der Aufgabe}

\subsection{Verbindungsaufbau}
Der erste Schritt beim Erstellen des Programms befasste sich mit dem Erstellen
einer Verbindung zur PostgreSQl-Datenbank. Dazu wurde die Klasse
DatabaseConnection, sowie ConnectionInformation implementiert. Die Klasse
ConnectionInformation dient der Erstellung eines gleichnamigen Objektes,
welches alle Informationen zur Datenbankanbindung enthält und an das Objekt vom
Typ DatabaseConnection  übergeben wird. Mit Hilfe der Funktion connect(), des
Objekts DatabaseConnection, wird eine Verbindung erzeugt, welche die
übergebenen Parameter verwendet.

\subsection{Erstellen der Abfragen}
Die Erstellung der Abfragen findet in der Klasse TXHandler statt, welche ein
gleichnamiges Objekt erzeugt. Das entstehende Objekt kann die Funktion
kontostandTX aufrufen, welche den Kontostand abfragt und zurückgibt. Es kann
außerdem mit Hilfe der Funktion einzahlungTX eine Einzahlung vornehmen, welche
den aktuellen Kontostand zurückgibt. Des Weiteren kann mit der Funktion
analyseTX nach der Häufigkeit der Einzahlung eines bestimmten Betrags gesucht
werden, wobei die Häufigkeit zurückgegeben wird.

\subsection{Erstellung der Programm-Schritte}
Die oben beschriebenen Abfragemöglichkeiten werden in den vorgegebenen
Schritten Einschwingphase, Messphase und Auss chwingphase in einer 35
(kontostandTX) zu 50 (einzahlungTX) zu 15 (analyseTX) ausgeführt. Die Verteilung
nach diesem Verhältnis übernimmt dabei die Funktion choose(). Das Aufrufen der
jeweiligen Funktion mit zufällig gewählten Parametern übernimmt die Funktion
doTX(). Wobei die zufälligen Werte durch die Funktion chooseNumber() bestimmt
werden, diese Funktion achtet dabei darauf, dass keine unzulässigen Werte
abgefragt werden.

Die Einschwingphase wird in der Funktion attackTime() beschrieben und dauert
wie in den Vorgaben beschrieben genau vier Minuten, wobei nach jeder Abfrage
eine Pause von 50ms eingelegt wird. Die Messphase wird als Funktion
benchStage() dargestellt und dauert genau fünf Minuten, wobei auch jeweils nach
jeder Abfrage eine Pause von 50ms eingelegt wird. Die Ausschwingphase wird in
der Funktion boomOutStage() dargestellt und dauert gemäß der Anforderungen eine
Minute, wobei nach jeder Abfrage 50ms Pause eingehalten werden.

\subsection{Erstellung der Threads}

Die erstellung der Threads zum mehrfachen starten des Load-Drivers findet in der
Main Methode statt. Dabei werden Funktionen aus der Klasse ClientThread bzw.
des erzeugten Objekts aufgerufen. In der Funktion run() erfolgt zudem das
Aufrufen der Programmschritte, diese werden je Thread aufgerufen.

\subsection{Exception-Handling}


\clearpage
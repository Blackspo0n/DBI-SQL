\section{Implementierung der Aufgabe}

\subsection{Verbindungsaufbau}
Der erste Schritt beim Erstellen des Programms befasste sich mit dem Erstellen
einer Verbindung zur PostgreSQl-Datenbank. 

Dazu wurde die Klasse \textbf{ConnectionInformation}, sowie
\textbf{ConnectionInformation} aus dem vorherigen Projekt implementiert. 

Die Klasse \textbf{ConnectionInformation} dient der Erstellung eines gleichnamigen
Objektes, welches alle Informationen zur Datenbankanbindung enthält und an das
Objekt vom Typ \textbf{DatabaseConnection} übergeben wird. Mit Hilfe der Funktion
\textbf{connect()}, des Objekts \textbf{DatabaseConnection}, wird eine Verbindung
erzeugt, welche die übergebenen Parameter verwendet.

\subsection{Erstellen der Abfragen}
Die Erstellung der Abfragen findet in der Klasse \textbf{TXHandler} statt, welche
ein gleichnamiges Objekt erzeugt. 

Das entstehende Objekt kann die Funktion \textbf{kontostandTX} aufrufen, welche
den Kontostand abfragt und zurückgibt. Es ist ebenfalls Möglich mit Hilfe der
Funktion \textbf{einzahlungsTX} eine Einzahlung vornehmen. 

\lstset{language=Java, backgroundcolor=\color{editorGray},
  basicstyle={\linespread{0.82}\footnotesize\ttfamily},   
  frame=b, xleftmargin={0.75cm},literate=
    {Ö}{{\"O}}1
  {Ä}{{\"A}}1
  {Ü}{{\"U}}1
  {ß}{{\ss}}2
  {ü}{{\"u}}1
  {ä}{{\"a}}1
  {ö}{{\"o}}1,
    numberstyle=\tiny\noncopynumber,
      }
\begin{lstlisting}[caption={einzahlungsTX-Funktion in TXHandler.java}]
	/**
	 * Ermittlung eiens neuen Kontostandes.
	 * @return Neuer errechneter Kontostand.
	 * @throws SQLException Fehler bei der Datenbank Abfrage, mit spezifizierter Fehlermeldung.
	 */
	public double einzahlungTX(int accID, int tellerID, int branchID, double delta) throws SQLException	{
		ResultSet rs = null;
		try {
			connection.databaseLink.setAutoCommit(false);
			statement.executeUpdate("UPDATE branches SET balance=balance+"+delta+" WHERE branchid="+branchID);
			statement.executeUpdate("UPDATE tellers SET balance=balance+"+delta+" WHERE tellerid="+tellerID);
			statement.executeUpdate("UPDATE accounts SET balance=balance+"+delta+" WHERE accid="+accID);
			statement.executeUpdate("INSERT INTO history (accid, tellerid, delta, branchid, accbalance, cmmnt)"+ "VALUES("+accID+", "+tellerID+", "+delta+", "+branchID+", (SELECT balance FROM accounts WHERE accid="+accID+"), ' ')");
			rs = statement.executeQuery("SELECT balance FROM accounts WHERE accid = " + accID);
			connection.databaseLink.commit();
			if(rs.next() == false) throw new SQLException("Kontonummer nicht vorhanden");
		}
		catch (PSQLException e) {
			System.out.println(e.getMessage());
			connection.databaseLink.rollback();
		}
		finally {
			connection.databaseLink.setAutoCommit(true);
		}
		return rs.getDouble(1);
	}
\end{lstlisting}
 

Des Weiteren kann mit der Funktion \textbf{analyseTX} nach der Häufigkeit der
Einzahlung eines bestimmten Betrags gesucht werden, wobei die Häufigkeit zurückgegeben wird.

\subsection{Erstellung der Programm-Schritte}
Die oben beschriebenen Abfrage-Möglichkeiten werden in den vorgegebenen
Schritten Einschwingphase, Messphase und Ausschwingphase in einer Gwichtung von
35/50/15 ausgeführt.

Die Verteilung nach diesem Verhältnis übernimmt dabei die Funktion
\textbf{choose()}.

\begin{lstlisting}[caption={choose-Funktion in ProgramStage.java}]
	/**
	 * Entscheidet per gewichteten Zufall darüber, welche Abfrage an die Datenbank geschickt wird..
	 * 
	 * @return Gibt die Zahl 1, 2 oder 3 zurück. Dadurch wird im folgenden über die Abfrage entschieden.
	 */
	public int choose() {
		int number=(int)(Math.random()*100);
		if(number <= 35) {
			return 1;
		}
		else if( number > 35 && number <= 85) {
			return 2;
		}
		
		return 3;
	}
\end{lstlisting}

Das Aufrufen der jeweiligen Funktion mit zufällig gewählten Parametern
übernimmt die Funktion \textbf{doTX()}. Wobei die zufälligen Werte durch die
Funktion \textbf{chooseNumber()} bestimmt werden, diese Funktion achtet dabei
darauf, dass keine unzulässigen Werte abgefragt werden.


\begin{lstlisting}[caption={doTX-Funktion in ProgramStage.java}]
	/**
	 * Entscheidet über die auszuführende Abfrage.
	 * 
	 * @throws SQLException Fehler, welcher bei der Abfrage an die Datenbank auftreten kann.
	 */
	private void doTX() throws SQLException
	{
		switch(choose())
		{
			case 1:
				txh.kontostandTX(chooseNumber("accounts", 100));
				break;
			case 2:
				txh.einzahlungTX(
						chooseNumber("accounts", 100),
						chooseNumber("tellers", 100),
						chooseNumber("branches", 100),
						chooseNumber("delta", 100));
				break;
			case 3:
				txh.analyseTX(chooseNumber("delta", 100));
				break;
		}
	}
\end{lstlisting}

Die Einschwingphase wird in der Funktion \textbf{attackStage()} implementiert
und dauert wie in den Vorgaben beschrieben genau vier Minuten.

\clearpage

Die Messphase wird als Funktion \textbf{benchStage()} implementiert und dauert
genau fünf Minuten. Folgendermaßen wird diese implementiert:
\begin{lstlisting}[caption={benchStage-Funktion in ProgramStage.java}]
	/**
	 * Ausführen des zweiten Programmteils.
	 * 
	 * @throws SQLException Möglicher Fehler bei der DB-Abfrage.
	 * @throws InterruptedException Möglicher Fehler bei der Wartezeit.
	 */
	public void benchStage() throws SQLException, InterruptedException {
		int txCounter = 0;
		long start = System.currentTimeMillis();
		
		
		while(start + 300000 >= System.currentTimeMillis()) {
			doTX();
			txCounter++;
			Thread.sleep(50);	
		}
		
		Main.setTxCountSum(Main.getTxCountSum() + txCounter);
		
		System.out.println("Anzahl der Transaktionen: " + txCounter);
		System.out.println("Anzahl der Transaktionen pro Sekunde: " + (double)((double)txCounter/(double)300));
	}
\end{lstlisting}

Die Ausschwingphase wird in der Funktion \textbf{boomOutStage()} dargestellt und
dauert gemäß der Anforderungen eine Minute.

In jeder Funktion wird die \gqq{thinkTime} von 50 Millisekunden berücksichtigt.

\subsection{Erstellung der Threads}

Die Threads, welche jeweils einen Lastclient bereitstellt wird in der Klass
\textbf{ClientThread} implementiert. Diese erbt von der Klasse Thread und
implemtiert ein \textbf{run}-Funktion.

\begin{lstlisting}[caption={Ausschnitt aus ClientThread}]
/**
 *  Zuweisungen an Threads.
 *  @author Mario Kellner
 *	@author Markus Hausmann
 *  @author Jonas Stadtler
 */
public class ClientThread extends Thread {
\end{lstlisting}

Die Instanziierung der Threads finden anschließend in der
\textbf{Main}-Methode des Programms statt. Nach der Erstellung der Threads
soll mittels eines \textbf{Timestamps} auf ein Startsignal gewartet werden.


Das Startsignal wird gegeben, wenn der zuvor festgelegte \textbf{timestamp}
überschritten worden ist:

\begin{lstlisting}[caption={Wartezeit}]
		    System.out.println("Wartezeit für die Threads beträgt " + (time - System.currentTimeMillis()) + " Millisekunden");
		    while(time > System.currentTimeMillis()) {
		    	Thread.sleep(1);
		    }
		    
		    System.out.println("Wartezeit erreicht");
		    ClientThread.setStartTrans();
\end{lstlisting}


\subsection{Exception-Handling}

Das Exceptionhandlung findet global in der \textbf{Main}-Methode statt. Tritt
ein Fehler auf, wird dieser mittels dem \textbf{Throws}-Statement
\gqq{hochgebubbelt} bzw. weitergegeben bis dieser in unserer \textbf{Catch}-Klausel behandelt wird.

\begin{lstlisting}[caption={Catch-Block}]
		catch (SQLException sqle){
			System.out.println("Datenbank Fehler: " + sqle.getMessage());
			sqle.printStackTrace();
		}
 		catch (Exception e) {
			System.out.println("Ein Fehler ist aufgetreten: " + e.getMessage());
			e.printStackTrace();
		}
\end{lstlisting}

Da die \textbf{run}-Methode der Klasse \textbf{Thread} kein \textbf{throws}
unterstützt würde die Fehlerbehandlung für die Threads gesondert implementiert.
Zusätzlich wird von den Threads die Funktion \textbf{abortProgramm}
ausgeführt, wenn ein unbehandelter fehler im Thread auftreten sollte. Dies führt
dazu, dass dsa Programm automatisch die Ausführung des Programmcodes
unterbricht.

\clearpage
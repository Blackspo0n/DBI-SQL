\section{Implementierung der Aufgabe}

\subsection{Verbindungsaufbau}
Der erste Schritt beim Erstellen des Programms befasste sich mit dem Erstellen
einer Verbindung zur PostgreSQl-Datenbank. 

Dazu wurde die Klasse \textbf{ConnectionInformation}, sowie
\textbf{ConnectionInformation} aus dem vorherigen Projekt implementiert. 

Die Klasse \textbf{ConnectionInformation} dient der Erstellung eines gleichnamigen
Objektes, welches alle Informationen zur Datenbankanbindung enthält und an das
Objekt vom Typ \textbf{DatabaseConnection} übergeben wird. Mit Hilfe der Funktion
\textbf{connect()}, des Objekts \textbf{DatabaseConnection}, wird eine Verbindung
erzeugt, welche die übergebenen Parameter verwendet.

\subsection{Erstellen der Abfragen}
Die Erstellung der Abfragen findet in der Klasse \textbf{TXHandler} statt, welche
ein gleichnamiges Objekt erzeugt. 

Das entstehende Objekt kann die Funktion \gqq{kontostandTX} aufrufen, welche den
Kontostand abfragt und zurückgibt.Es kann außerdem mit Hilfe der Funktion
einzahlungTX eine Einzahlung vornehmen, welche den aktuellen Kontostand zurückgibt.


---------------- code hier einfügen

Des Weiteren kann mit der Funktion \textbf{analyseTX} nach der Häufigkeit der
Einzahlung eines bestimmten Betrags gesucht werden, wobei die Häufigkeit zurückgegeben wird.

\subsection{Erstellung der Programm-Schritte}
Die oben beschriebenen Abfrage-Möglichkeiten werden in den vorgegebenen
Schritten Einschwingphase, Messphase und Ausschwingphase in einer Gwichtung von
35/50/15 ausgeführt.

Die Verteilung nach diesem Verhältnis übernimmt dabei die Funktion
\gqq{choose()}.

--- code verteilungsfunktion

Das Aufrufen der jeweiligen Funktion mit zufällig gewählten Parametern übernimmt die Funktion
doTX(). Wobei die zufälligen Werte durch die Funktion chooseNumber() bestimmt
werden, diese Funktion achtet dabei darauf, dass keine unzulässigen Werte
abgefragt werden.


-- code doTX

Die Einschwingphase wird in der Funktion \textbf{attackStage()} implementiert
und dauert wie in den Vorgaben beschrieben genau vier Minuten.

Die Messphase wird als Funktion \textbf{benchStage()} implementiert und dauert
genau fünf Minuten.

Die Ausschwingphase wird in der Funktion \textbf{boomOutStage()} dargestellt und
dauert gemäß der Anforderungen eine Minute.

In jeder Funktion wird die \gqq{thinkTime} von 50 Millisekunden berücksichtigt.

\subsection{Erstellung der Threads}

Die Threads, welche jeweils einen Lastclient bereitstellt wird in der Klass
\gqq{ClientThread} implementiert. Diese erbt von der Klasse Thread und
implemtiert ein \textbf{run}-Funktion.

----- code ausschnitt clientthread

Die Instanziierung der Threads finden anschließend in der
\textbf{main}-Methode des Programms statt. Nach der Erstellung der Threads
soll mittels eines \textbf{timestamps} auf ein Startsignal gewartet werden.

Das Startsignal wird gegeben, wenn der zuvor festgelegte \textbf{timestamp}
überschritten worden ist:

----------- protgrammcode von der schleife


\subsection{Exception-Handling}

Das Exceptionhandlung findet global in der \textbf{main}-Methode statt. Tritt
ein Fehler auf, wird dieser mittels dem \textbf{throws}-Statement
\gqq{hochgebubbelt} bis dieser in unserer \textbf{catch}-Klausel behandelt wird.

---- code vom catchblock

Da die \textbf{run}-Methode der Klasse \textbf{Thread} kein \textbf{throws}
unterstützt würde die fehlerbehandlung für die Threads gesondert implementiert.
Zusätzlich wird von den Threads die Funktion \textbf{abortProgramm}
ausgeführt, wenn ein unbehandelter fehler im Thread auftreten sollte. Dies führt
dazu, dass dsa Programm automatisch die Ausführung des Programmcodes
unterbricht.

\clearpage
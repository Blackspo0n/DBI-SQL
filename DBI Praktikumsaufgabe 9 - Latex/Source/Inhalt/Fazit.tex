\section{Fazit zu PostgreSQL}
Nach mehrmonatiger Benutzung des Datenbankmanagement-Systems (DBMS) PostgreSQL für 
unterschiedlichste Aufgaben, ist dieses DBMS unserer Meinung nach eine gute Alternative zu MySQL 
und ähnlichem. 

Denn für PostgreSQL existieren alle nötigen JDBC-Treiber und Anbindungsmöglichkeiten, die benötigt 
werden. Zudem gibt es viele Informationen, Tutorials und auch Bücher dazu, die meisten in englischer 
Sprache sind. Weiter existieren ebenfalls etliche Forenbeiträge bzw.
Hilfestellungen im Internet die einen bei der Installation und Verwendung des
Systems sehr unterstützen.

PostgreSQL implementiert den SQL-Standard, so ist es möglich es an Programme wie
Microsoft Access anzubinden. Es bietet viele Einstellungsmöglichkeiten, mit
welchen man die Einstellung auf die Anforderungen anpassen kann, um eine gute
Performance zu erreichen. Zur Optimierung findet man viele Tipps und
Orientierungswerte, nach nur kurzer Suche, im Internet.

Wir hatten bisher keine Probleme mit PostgreSQL, es ist nie abgestürzt oder hat eine Form von 
Datenverlust gezeigt. Im Allgemeinen wirk ProsgreSQL sehr robust.

Die einzige Schwäche von PostgreSQL ist das mitgelieferte Tool pgAdmin4, welches
anfänglich sehr unübersichtlich gestaltet war, anders als PhpMyAdmin beispielsweise. Besonders
das Erstellen von Tabellen und Benutzern erweist sich mit dem Tool als sehr
unübersichtlich, zudem ist es scheinbar nicht möglich Userinformationen der
Serververbindung nachträglich zu ändern. Nach einiger Zeit lernt man aber auch
mit diesem Tool zu arbeiten, wobei das meist benutzte Werkzeug in diesem
Programm das soganntente \textbf{Querytool} war.
   
Auf Grund dieser Erfahrungen würden wir das DBMS PostgreSQL als eine gute
Alternative zu MySQL und ähnlichem ansehen. Ansich richtet sich PostgreSQL eher
an Benutzer, welche bereits fundierte Kentnisse über solche Systemen
besitzen. Trotzdem kann man es für alle wichtigen Anwendungsbereiche
verwenden und es bietet dem Benutzer viele Möglichkeiten das System auf sich
und seine Bedürfnisse anzupassen, ohne großen Aufwand.


\clearpage
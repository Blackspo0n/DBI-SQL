\section{Fazit zu PostgreSQL}
Nach mehrmonatiger Benutzung des Datenbankmanagement-Systems (DBMS) PostgreSQL für 
unterschiedlichste Aufgaben, ist dieses DBMS unserer Meinung nach eine gute Alternative zu MySQL 
und ähnlichem. 

Denn für PostgreSQL existieren alle nötigen JDBC-Treiber und Anbindungsmöglichkeiten, die benötigt 
werden. Zudem gibt es viele Informationen, Tutorials und auch Bücher dazu, die meisten in englischer 
Sprache sind. Es gibt auch viele Forenbeiträge bzw. Hilfestellungen im Internet,
falls es ein Problem bei der Verwendung oder der Installation gibt.

PostgreSQL unterstützt SQL, man kann es aber auch an Programme wie Microsoft Access anbinden. Es 
bietet viele Einstellungsmöglichkeiten, mit welchen man die Einstellung auf die Anforderungen 
anpassen kann, um eine gute Performance zu erreichen. Zur Optimierung findet man viele Tipps und 
Orientierungswerte, nach nur kurzer Suche, im Internet.

Wir hatten bisher keine Probleme mit PostgreSQL, es ist nie abgestürzt oder hat eine Form von 
Datenverlust gezeigt, das DBMS wirkt somit sehr robust.

Die einzige Schwäche von PostgreSQL ist das mitgelieferte Tool PgAdmin, welches anfänglich sehr 
unübersichtlich gestaltet ist, anders als PhpMyAdmin beispielsweise. Besonders das Erstellen von 
Tabellen und Benutzern erweist sich mit dem Tool als sehr übersichtlich, zudem ist es scheinbar 
nicht möglich das Passwort nachträglich zu ändern. Nach einiger Zeit lernt man aber auch mit diesem 
Tool zu arbeiten, wobei das meist benutzte Werkzeug das Querytool war.
   
Auf Grund dieser Erfahrungen würden wir das DBMS PostgreSQL als gute Alternative zu MySQL und 
ähnlichem sehen, wobei sich PostgreSQL eher an Benutzer richtet, welche bereits etwas Erfahrung mit 
solchen Systemen mitbringen. Trotzdem kann man es für alle wichtigen Anwendungsbereiche verwenden 
und es bietet dem Benutzer viele Möglichkeiten das System auf sich und seine Bedürfnisse anzupassen, 
ohne großen Aufwand.


\clearpage
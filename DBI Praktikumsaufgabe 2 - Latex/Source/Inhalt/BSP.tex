\section{Analysephase}
\label{sec:Analysephase}

\subsection{SQL Test}
\label{sec:SQL Test}
\lstset{language=SQL,numbers=none}
Das ist ein PostgreSQL Test: \\
\begin{lstlisting}
SELECT p.pid FROM orders AS o 
INNER JOIN products AS p ON (o.pid = p.pid)
INNER JOIN agents AS a ON (o.aid = a.aid)
INNER JOIN costumers AS c ON (o.cid = c.cid)
WHERE a.city = 'Tokio' AND c.city = 'Dallas'
\end{lstlisting}

\subsection{IST-Analyse}
\label{sec:IstAnalyse}

Fast jedes von der \acl{mention} geschriebene .NET-Programm besitzt ein eigenes
Rechtesystem, welches bestimmte Programmfunktionen in der Benutzung einschränkt.

Die Zugriffsrechte der meisten .NET-Programme werden unabhängig von der
\acs{WAWI}, meist in verschlüsselten \gqq{.dat} Dateien, gespeichert. Diese
Rechte sind nur im Programm selber gültig und haben keinerlei Einfluss auf
andere Anwendungen, da diese im Programmverzeichnis abgelegt werden.

In der Tabelle \Reference{tab:verschiedeneRechteverwaltungen} ist eine
Übersicht über aktuell gewartete Programme und deren eigenständigen
Rechtesysteme gegeben.\\

\tabelle{verschiedene
Rechteverwaltungen}{tab:verschiedeneRechteverwaltungen}{ZeitplanungKomplett.tex}

\subsection{Wirtschaftlichkeitsanalyse}
\label{sec:Wirtschaftlichkeitsanalyse}
Durch die vielen einzelnen Implementationen sind in den letzten Monaten
immer wieder Schwierigkeiten bei der Rechtevergabe aufgetreten. Um dieses
Problem zu lösen wurde unter Umständen bis zu ein Entwicklertag
investiert.

Durch die Entwicklung einer zentralen Rechteverwaltung für die .NET
Programme für \ac{mention} soll dieses immer wiederkehrende Problem
gelöst werden.
Durch diese Lösung sollen dem Unternehmen \ac{mention}
keine unnötigen Kosten mehr entstehen und die Kosten für künftige Projekte
gering gehalten werden.\\

\subsubsection{\gqq{Make or Buy}-Entscheidung}
\label{sec:MakeOrBuyEntscheidung}
Um zu ermitteln wie das Problem mit der Rechteverwaltung zu lösen ist wurde,
bevor die Entwurfsphase begonnen wurde, nach einer bereits fertigen und kostengünstigen Lösung gesucht.

Eine fertige Lösung zur Verwaltung der Benutzerrechte müsste die folgenden
Kriterien besitzen damit sie für die Zwecke des Betriebs \ac{mention} in Frage
kommt:
\begin{itemize}
	\item Kompatibilität zur \ac{MSSQL} Datenbank der \ac{mention} \ac{WAWI}
	\item gegliedert in Nutzer, Rollen, Berechtigungen
	\item Backend zur Administration der Nutzerrechte
	\item integrierbar als Programmbibliothek in aktuelle .NET-Programme
\end{itemize}

Durch den wichtigsten Punkt, die Datenbank des \acs{WAWI}, ist eine eigene
Implementation die beste Alternative, denn für etwaige Lösungen müsste die
\ac{WAWI} angepasst werden, was wiederum einen nicht schätzbaren Aufwand
darstellen würde, da jedes Programm, einschließlich der FoxPro Programme,
dahingehend geändert werden müsste.

Recherchiert wurde die im Zend
Framework\footnote{\url{http://framework.zend.com/}} befindliche ACL
Technologie. Das Akronym ACL steht für Access Control List. Würde diese Lösung
verwendet werden, ständen Nutzerrollen und Zugriffsregeln zur Verfügung. Das
Framework basiert jedoch auf der Programmiersprache
PHP\footnote{\url{http:///www.php.net/}}. Diese Sprache ist mit Standardmitteln
nicht als eine \ac{DLL} kompilierbar.

Für die Programmiersprache C\# existiert keine eigenständige Lösung zum
Administrieren der Rechte von Benutzern.\\

\subsubsection{Projektkosten}
\label{sec:Projektkosten}

Die Entwicklungskosten für das Projekt die durch die Durchführung des
Projektes entstehen setzen sich sowohl aus Personal-, als auch aus
Ressourcenkosten zusammen.

Der gemittelte Stundensatz der festangestellten Entwickler beläuft sich auf
\eur{50}.\newline Der monatliche Bruttolohn in dem dritten Lehrjahr von dem
Autor und Auszubildenden \autor{}, liegt bei \eur{750}.
Der Stundensatz von Autor \autor{} ist allerdings nicht
festgelegt. Um aber dennoch die Kosten korrekt darzulegen hat der Autor \autor{}
den Stundensatz mittels nachstehender Berechnung errechnet:

\begin{eqnarray}
8 \mbox{ h/Tag} \cdot 220 \mbox{ Tage/Jahr} = 1760 \mbox{ h/Jahr}\\
\eur{750}\mbox{/Monat} \cdot 12 \mbox{ Monate/Jahr} = \eur{9000}
\mbox{/Jahr}\\
\frac{\eur{9000} \mbox{/Jahr}}{1760 \mbox{ h/Jahr}} \approx \eur{5.12}\mbox{/h}
\end{eqnarray}
\setcounter{equation}{0}

Es ergibt sich ein Stundenlohn von \eur{5.12}. 

Die Durchführungszeit des Projektes beträgt 70 Stunden. Für die Nutzung von
Ressourcen\footnote{Räumlichkeiten, Arbeitsplatzrechner etc.} wird ein
pauschaler Stundensatz von \eur{15} angenommen.
Eine Aufstellung der Kosten befindet sich in
der Tabelle. 

Insgesamt wurden, unter der Berücksichtigung der genannten
Wertestellungen, Kosten in Höhe von \eur{1723,40} errechnet.

\subsubsection{Amortisationsdauer / Gewinnbringende Nutzung}
\label{sec:Amortisationsdauer}

Durch die Entwicklung dieser Bibliothek fallen Kosten an. Durch
nachstehende Rechnung soll die Amortisationsdauer\footnote{Zeit für die
Zurückgewinnung der Projektkosten} berechnet werden und mit dem IST-Zustand,
der im Abschnitt \Reference{sec:IstAnalyse} erläutert wird, verglichen werden.

\begin{itemize}
	\item Geschätzter Aufwand für die Entwicklung eigener Systeme: 4
	Entwicklerstunden (\eur{50}/h)
	\item Geschätzter Aufwand für die Implementation der Bibliothek: 30 Minuten
	(\eur{50}/h)
\end{itemize}

Die gesparten Kosten pro Projekt ergeben sich anschließend wie
folgt:\begin{eqnarray} (4\mbox{h} - 0.5\mbox{h}) \cdot \eur{50} = \eur{175}
\end{eqnarray}
\setcounter{equation}{0}

Für die Amortisationsdauer ergibt sich folgende Rechnung:\begin{eqnarray}
\eur{175} = \mbox{gesparte Kosten/Projekt}\\
\frac{\mbox{Projektkosten}}{\mbox{gesparte Kosten}} = 
\frac{\eur{1723,40}}{\eur{175}}
= 9,848\mbox{ Projekte} \approx 10\mbox{ Projekte}
\end{eqnarray}
\setcounter{equation}{0}

Die Amortisationsdauer des Projektes liegt also bei 9
Projektimplementierungen.\\

\subsection{Lastenheft}
\label{sec:Lastenheft}
Um die konkreten Anforderungen, die an das Projekt gestellt werden, festzuhalten
wurde ein Lastenheft von dem Ausbilder und Auftraggeber \ac{mention} erstellt.

In diesem wurden die Projektanforderungen aus der Sicht des Auftragsgebers
definiert. Das gekürzte Lastenheft befindet sich im \Anhang{app:Lastenheft}.

\addsec{Praktikumsaufgabe 2)}
\label{sec:Praktikumsaufgabe 2}
\lstset{language=SQL,numbers=none}

\subsection{Teilaufgabe a)}
\label{sec:Teilaufgabe a}
Bestimme alle Paare aus ORDNO und PID für Bestellungen mit einer QTY größer oder
gleich 1000.
\begin{lstlisting}
SELECT ordno, 
       pid 
FROM   orders 
WHERE  qty >= 1000
\end{lstlisting}
Anzahl der Ergebnistupel: 7

\subsection{Teilaufgabe b)}
\label{sec:Teilaufgabe b}
Bestimme alle Produktnamen für Produkte mit einem Preis zwischen 0.50 und 1.00
Dollar inklusive beider Grenzen.

\begin{lstlisting}
SELECT pname 
FROM   products 
WHERE  price >= 0.5 
       AND price <= 1
\end{lstlisting}
Anzahl der Ergebnismenge: 6


\subsection{Teilaufgabe c)}
\label{sec:Teilaufgabe c}
Bestimme alle Paare aus ORDNO und CNAME für Bestellungen mit einem Wert unter 500 Dollar.

\begin{lstlisting}
SELECT ordno, 
       cname 
FROM   orders AS o 
       INNER JOIN customers 
               ON o.cid = customers.cid 
WHERE  dollars < 500 
\end{lstlisting}
Anzahl der Ergebnistupel: 6


\subsection{Teilaufgabe d)}
\label{sec:Teilaufgabe d}
Bestimme alle Trippel aus ORDNO, CNAME und ANAME für Bestellungen im März.

\begin{lstlisting}
SELECT o.cid, 
       o.aid, 
       o.month 
FROM   orders AS o 
       INNER JOIN customers 
               ON ( o.cid = customers.cid ) 
       INNER JOIN agents 
               ON ( o.aid = agents.aid ) 
WHERE  o.month = 'mar'

\end{lstlisting}
Anzahl der Ergebnistripel: 3


\subsection{Teilaufgabe e)}
\label{sec:Teilaufgabe e}
Bestimme die Produktnamen der Produkte, die in Duluth gelagert werden und für die im März Bestellungen vorlagen.

\begin{lstlisting}
SELECT pname, 
       p.pid, 
       city 
FROM   products AS p 
       LEFT JOIN orders AS o 
              ON ( p.pid = o.pid ) 
WHERE  p.city = 'Duluth' 
       AND o.month = 'mar' 

\end{lstlisting}
Anzahl der Ergebnistripel: 0


\subsection{Teilaufgabe f)}
\label{sec:Teilaufgabe f}
Bestimme alle möglichen Trippel aus CID, AID und PID, so dass alle dieselbe Adresse besitzen. 

\begin{lstlisting}
SELECT c.cid, 
       a.aid, 
       p.pid 
FROM   customers AS c 
       INNER JOIN agents AS a 
               ON ( c.city = a.city ) 
       INNER JOIN products AS p 
               ON( a.city = p.city ) 
\end{lstlisting}
Anzahl der Ergebnistripel: 10


\subsection{Teilaufgabe g)}
\label{sec:Teilaufgabe g}
Bestimme alle möglichen Trippel aus CID, AID und PID, so dass nicht alle dieselbe Adresse besitzen (jeweils zwei Adressen dürfen aber übereinstimmen). 

\begin{lstlisting}
SELECT c.cid, 
       a.aid, 
       p.pid 
FROM   products AS p, 
       customers AS c, 
       agents AS a 
WHERE  ( p.city ! = c.city 
          OR c.city != a.city 
          OR a.city	 != p.city )  

\end{lstlisting}
Anzahl der Ergebnistripel: 200


\subsection{Teilaufgabe h)}
\label{sec:Teilaufgabe h}
Bestimme nochmals alle möglichen Trippel aus CID, AID und PID, so dass keine zwei Adressen übereinstimmen. 

\begin{lstlisting}
SELECT c.cid, 
       a.aid, 
       p.pid 
FROM   products AS p, 
       customers AS c, 
       agents AS a 
WHERE  ( p.city != c.city 
         AND c.city != a.city 
         AND a.city != p.city )

\end{lstlisting}
Anzahl der Ergebnistripel: 107

\subsection{Teilaufgabe i)}
\label{sec:Teilaufgabe i}
Bestimme die Produktnamen der Produkte, die von mindestens einem Kunden aus Dallas durch einen Vertriebsagenten in Tokyo bestellt wurden.

\begin{lstlisting} 
SELECT p.pname 
FROM   orders AS o 
       INNER JOIN agents AS a 
               ON o.aid = a.aid 
       INNER JOIN products AS p 
               ON p.pid = o.pid 
       INNER JOIN customers AS c 
               ON o.cid = c.cid 
WHERE  a.city = 'Tokyo' 
       AND c.city = 'Dallas'

\end{lstlisting}
Anzahl der Ergebnismenge: 3


\subsection{Teilaufgabe j)}
\label{sec:Teilaufgabe j}
Bestimme die PIDs der Produkte, die durch Vertriebsagenten bestellt wurden, die mindestens eine Bestellung für einen Kunden aus Kyoto durchgeführt haben. 

\begin{lstlisting}
SELECT p.pid 
FROM   orders AS o 
       INNER JOIN agents AS a 
               ON o.aid = a.aid 
       INNER JOIN products AS p 
               ON p.pid = o.pid 
       INNER JOIN customers AS c 
               ON o.cid = c.cid 
WHERE  a.city = 'Kyoto' 
\end{lstlisting}
Anzahl der Ergebnismenge: 3


\subsection{Teilaufgabe k)}
\label{sec:Teilaufgabe k}
Bestimme AID-Paare, so dass die beiden unterschiedlichen(!) Agenten in derselben CITY beheimatet sind.

\begin{lstlisting}
SELECT a1.aid, 
       a2.aid 
FROM   agents AS a1 
       INNER JOIN agents AS a2 
               ON ( a1.aid != a2.aid 
                    AND a1.city = a2.city )  

\end{lstlisting}
Anzahl der Ergebnistupel: 2


\subsection{Teilaufgabe l)}
\label{sec:Teilaufgabe l}
Bestimme die Kundennummern der Kunden, die keine Bestellung über den Vertriebsagenten „a03“ getätigt haben. 

\begin{lstlisting}
SELECT DISTINCT cid 
FROM   orders 
       INNER JOIN agents 
               ON ( orders.aid = agents.aid ) 
WHERE  agents.aid != 'a03'  
\end{lstlisting}
Anzahl der Ergebnistmenge: 4


\subsection{Teilaufgabe m)}
\label{sec:Teilaufgabe m}
Bestimme die Kundennummern der Kunden, die sowohl das Produkt „p01“ als auch das Produkt „p07“ bestellt haben. 

\begin{lstlisting}
SELECT DISTINCT cid 
FROM   orders 
WHERE  cid IN (SELECT cid 
               FROM   orders 
               WHERE  pid = 'p07' 
                       OR pid = 'p01')  

\end{lstlisting}
Anzahl der Ergebnistripel: 3

\subsection{Teilaufgabe n)}
\label{sec:Teilaufgabe n}
Bestimme die Kundennummern der Kunden, die bereits alle Produkte bestellt haben. 

\begin{lstlisting}
SELECT cid 
FROM   customers 
WHERE  (SELECT Count(DISTINCT pid) 
        FROM   orders 
        WHERE  customers.cid = orders.cid) = (SELECT Count(pid) 
                                              FROM   products)  

\end{lstlisting}
Anzahl der Ergebnismenge: 1

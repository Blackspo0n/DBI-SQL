\section{Anhang}
\subsection{Programm (optimiert)}
\label{app:programmv2}

\begin{lstlisting}[caption={Main (optimiert)}, label={lst:mainv2}]
/**
 * 
 */
package de.whs.dbi.pa7;

import de.whs.dbi.pa7.database.ConnectionInformation;
import de.whs.dbi.pa7.database.DatabaseConnection;
import de.whs.dbi.pa7.database.TpsCreator;
import de.whs.dbi.pa7.database.TpsCreatorInterface;
import de.whs.dbi.pa7.database.TpsCreatorOldStatements;

/**
 * Unsere Hauptklasse. Hauptsächlich dient Sie als Einsteigspunkt unserer Applikation
 * und bietet gleichzeitig dem Benutzer eine Interaktionsmöglichkeit
 * über die Konsole an.
 * 
 * @author Mario Kellner
 * @author Markus Hausmann
 * @author Jonas Stadtler
 *
 */
public class Main {
	/**
	 * Datenbank Verbindungsobjekt
	 */
	static DatabaseConnection con;
	
	/**
	 * Ersteller der Datenbank
	 */
	static TpsCreatorInterface tpsDBCreator;
	
	/**
	 * Vom Benutzer angegebene Größe der tps-Datenbank
	 */
	static int sizeN;
	
	/**
	 * Gibt an, ob die heardcodierten Verbindungsinformationen benutzt werden.
	 */
	static boolean useLocal = false;
	
	/**
	 * Unserer Einstiegspunkt füt unsere Anwendung.
	 * Hier werden die nötigen Informationen von dem Benutzer abgefragt
	 * und der Benchmark Klasse übergeben.
	 * 
	 * 
	 * @param args Wird irgoriert
	 */
	
	public static void main(String[] args) {
		
		System.out.println("Lokale Datenbank verwenden? [j/n]:");
		useLocal = ConsoleReader.readJn();
		
		ConnectionInformation infos = new ConnectionInformation();
		
		if(useLocal) {
			// Lokale Informationen unserer Gruppe.
			// Diese Zugangsdaten werden nicht bei einer anderen Gruppe funktionieren.
			infos.setHost("127.0.0.1");
			//infos.setHost("192.168.122.70");
			infos.setDatabase("benchmark");
			infos.setUser("postgres");
			infos.setPassword("DBIPr");
		}
		else {
			
			//Verbindungnsinformationen werden vom User angegeben.
			System.out.println("Verbindungsinformationen eingeben!");
			
			System.out.println("Host:");
			infos.setHost(ConsoleReader.readString());

			System.out.println("Datenbank:");
			infos.setDatabase(ConsoleReader.readString());
			
			System.out.println("Benutzer:");
			infos.setUser(ConsoleReader.readString());
			
			System.out.println("Password:");
			infos.setPassword(ConsoleReader.readString());
		}
		

		try {
			// erstelle die Datenbankverbindung und verbinde
			System.out.print("Verbinde zur Datenbank ... ");
			
			con = new DatabaseConnection(infos);
			con.connect();
			
			System.out.println("[OK]");
			
			// Nutzer fragen, ob er die unoptimierte Programmversion nutzen möchte
			System.out.println("Optimierte Programmversion benutzen?? [j/n]:");
			
			// erstellen des jeweiligen TpsCreatorObjekts
			if(!ConsoleReader.readJn()) {
				tpsDBCreator = new TpsCreatorOldStatements(con);
			}
			else {

				tpsDBCreator = new TpsCreator(con);
			}
		}
		catch ( Exception err){
			
			//Fehlerfall
			err.printStackTrace();
			System.err.println(err.getMessage());
			System.err.println("Konnte keine Verbindung zur Datenbank aufbauen!");
			
			return;
		}
		
		
		// Nutzer die größe der Datenbank angeben lassen
		System.out.println("Gebe bitte die Größe (n) der Datenbank ein:");	
		sizeN = ConsoleReader.readInt();			
		
		// Zeige das Benchmarkmenü
		benchmarkMenu();
		
		System.out.println("Beende Anwendung...");		
	}
	
	/**
	 * Zeigt das Benchmarkmenü und wertet die Nutzereingaben
	 * aus.
	 */
	public static void benchmarkMenu() {
		int menureturn = renderMenu();
		
		// Übergabe der Benchmarkparameter
		switch(menureturn) {
		case 1:
			Benchmark.BenchmarkTask(sizeN, tpsDBCreator, true, false, true);
			break;
		case 2:
			Benchmark.BenchmarkTask(sizeN, tpsDBCreator, true, false, false);
			break;
		case 3:
			Benchmark.BenchmarkTask(sizeN, tpsDBCreator, true, true, true);
			break;
		case 4:
			Benchmark.BenchmarkTask(sizeN, tpsDBCreator, true, true, false);
			break;
		case 5:
			Benchmark.BenchmarkTask(sizeN, tpsDBCreator, false, false, true);
			break;
		case 6:
			Benchmark.BenchmarkTask(sizeN, tpsDBCreator, false, false, false);
			break;
		case 7:
			Benchmark.BenchmarkTask(sizeN, tpsDBCreator, false, true, true);
			break;
		case 8:
			Benchmark.BenchmarkTask(sizeN, tpsDBCreator, false, true, false);
			break;
		default:
			System.out.println("Ungültige Menü Option");
		}
	}
	
	
	/**
	 * Rendert das Benchmarkmenü.
	 * 
	 * @return Nutzereingabe
	 */
	public static int renderMenu() {
		System.out.println("======================- Benchmark Menu -===========================");
		System.out.println("= Wähle bitte eine der folgende Optionen:                         =");
		System.out.println("= (1) Benchmark, Debug Log, incl. drop & create                   =");
		System.out.println("= (2) Benchmark, Debug Log, excl. drop & create                   =");
		System.out.println("= (3) Benchmark, Debug Log, transactions, incl. drop & create     =");
		System.out.println("= (4) Benchmark, Debug Log, transactions, excl. drop & create     =");
		System.out.println("= (5) Benchmark, incl. drop & create                              =");
		System.out.println("= (6) Benchmark, excl. drop & create                              =");
		System.out.println("= (7) Benchmark, transactions, incl. drop & create                =");
		System.out.println("= (8) Benchmark, transactions, excl. drop & create                =");
		System.out.println("===================================================================");
		System.out.print("= Auswahl: ");
		
		return ConsoleReader.readInt();
	}
}
\end{lstlisting}

\begin{lstlisting}[caption={ConsoleReader (optimiert)}, label={lst:crv2}]
package de.whs.dbi.pa7;

import java.io.BufferedReader;
import java.io.IOException;
import java.io.InputStreamReader;

/**
 * Einige Konsolen Helfer. Die Klasse ist con der BasicIO.java aus dem GDI1 Kurs adaptiert
 * 
 * @author Mario Kellner
 * @author Markus Hausmann
 * @author Jonas Stadtler
 */
public class ConsoleReader {
	
	/**
	 * Liest ein j/J oder ein n/N aus der Konsole
	 * 
	 * @return j = true und n = false
	 */
	public static boolean readJn() {
		BufferedReader br = new BufferedReader(new InputStreamReader(System.in));
		
		try {
			String strTmp = br.readLine();
			if(strTmp.length() != 0 && (strTmp.charAt(0) == 'j' || strTmp.charAt(0) == 'J')) {
				return true;
			}
		}
		catch(Exception err) {}
		
		return false;
	}
	
	/**
	 * Liest einen String aus der Konsole aus
	 * 
	 * @return ausgelesener String
	 */
	public static String readString () {
		BufferedReader din = new BufferedReader(new InputStreamReader(System.in)); 
		
		try {	
			return din.readLine();
		}
		catch(IOException e) {				  
			return "Ein-/Ausgabe-Fehler";
		}
	}
	
	/**
	 * Liest einen Integer aus der Konsole aus
	 * 
	 * @return ausgelesener String
	 */
	public static int readInt()
	{
		try 
		{
			BufferedReader din = new BufferedReader(new InputStreamReader(System.in));
			return Integer.parseInt(din.readLine());
		}
		catch (NumberFormatException e)
		{
			System.out.println("Bitte gebe eine gültige Zahl ein.");
			return readInt();
		}
		catch (IOException e)
		{
			return -1;
		}
	}
}
\end{lstlisting}

\begin{lstlisting}[caption={Benchmark (optimiert)}, label={lst:bmv2}]
package de.whs.dbi.pa7;

import java.io.PrintStream;

import de.whs.dbi.pa7.database.TpsCreatorInterface;

public class Benchmark {
	
	/**
	 * Globale Startzeit des Benchmarks
	 */
	private static long startTime;
	
	/**
	 * Führt den Benchmark mit den Übergebenen Parametern durch.
	 * 
	 * @param size Die geplante Größe der n-tps-Datenbank
	 * @param tpsDBCreator Das Objekt, welches die Datenbank erzeugen kann
	 * @param debugLog Aktiviert das Debug Logging (SQL-Queries)
	 * @param useTransaction Gibt an, ob SQL-Transaktionen für die Datenbank ausgeführt werden sollen
	 * @param usePreamble Gibt an, ob die drop- und createstatements mit ins Benchmark mit einfließen sollen
	 */
	public static void BenchmarkTask(int size, TpsCreatorInterface tpsDBCreator, boolean debugLog, boolean useTransaction, boolean usePreamble) {
		tpsDBCreator.setDebug(debugLog);
		tpsDBCreator.createFiles(size);
		
		System.out.println("Brenchmarking gestartet");
		System.out.println("Parameter: debugLog = " + debugLog + ", useTransaction = " + useTransaction + ", usePreamble = " + usePreamble);

		
		if(!usePreamble) {
			startTimer();
			if(useTransaction) {
				tpsDBCreator.beginTransaktion();
			}
		}

		tpsDBCreator.dropSchema();
		tpsDBCreator.createSchema();
		
		if(usePreamble) {
			startTimer();
			if(useTransaction) {
				tpsDBCreator.beginTransaktion();
			}
		}
		
		long localTime  = System.currentTimeMillis();
		tpsDBCreator.createBranchTupel(size);
		System.out.println("Branches angelegt! Dauer " + (System.currentTimeMillis() - localTime) + " Millisekunden.");
		
		localTime  = System.currentTimeMillis();
		tpsDBCreator.createAccountTupel(size);
		System.out.println("Accounts angelegt! Dauer " + (System.currentTimeMillis() - localTime) + " Millisekunden.");
		
		localTime  = System.currentTimeMillis();
		tpsDBCreator.createTellerTupel(size);
		System.out.println("Teller angelegt! Dauer " + (System.currentTimeMillis() - localTime) + " Millisekunden.");
		
		tpsDBCreator.finishUp();
		
		if(useTransaction) {
			tpsDBCreator.endAndCommitTransaction();
		}
		
		System.out.println("Benchmark abgeschlossen. Gesamtdauer " + getCurrentMilliseconds() + " Millisekunden.");

		
	}
	
	/**
	 * Setzt die Startzeit auf die jetzige Zeit
	 */
	public static void startTimer() {
		startTime  = System.currentTimeMillis();
	}
	
	/**
	 * Gibt die Differenz aus der Startzeit und der jetzigen Zeit aus
	 * 
	 * @return Differenz der Subtraktion
	 */
	public static long getCurrentMilliseconds() {
		return System.currentTimeMillis() - startTime;
	}
}
\end{lstlisting}

\begin{lstlisting}[caption={DatabaseConnection (optimiert)}, label={lst:dbv2}]
package de.whs.dbi.pa7.database;

import java.sql.Connection;
import java.sql.DriverManager;
import java.sql.SQLException;

/**
 * Unsere Datebankklasse ermöglicht die Kommunikation
 * mit der Datenbank. Es ist ein Wrapper für das
 * Connection-Objekt der JDBC Connection
 * 
 * @author Mario Kellner
 * @author Markus Hausmann
 * @author Jonas Stadtler
 */
public class DatabaseConnection 
{
	/**
	 * Das eigentliche Connection Objekt
	 */
	public Connection databaseLink;
	
	/**
	 *  Unser Verbindungs-Model
	 */
	private ConnectionInformation ci;
	
	/**
	 * Der Konstruktur der Klasse.  Überprüft ob das Objekt im C
	 * 
	 * @param cInfo Das Connection Objekt
	 * @throws Exception Wird geworfen, wenn das Objekt was übergeben wird null ist.
	 */
	public DatabaseConnection(ConnectionInformation cInfo) throws Exception {
		if(cInfo == null) {
			throw new Exception("ConnectionInfo Object cannot be null");
		}
		
		ci = cInfo;
	}
	
	/**
	 * 
	 * List die Informatioen aus dem Objekt ci (ConnectionInformation) und versucht eine Verbindung 
	 * zur Datenbank auf zu bauen.
	 * 
	 * @throws SQLException Wird geworfen, wenn der DriverManager keine Verbndung zur Datenbank aufbauen kann
	 */
	public void connect() throws SQLException
	{
		/*
		 * Der Compiler übersetzt " + var + " automatisch in ein StringBuilder Objekt
		 */
		databaseLink = DriverManager.getConnection(
				"jdbc:postgresql://" + ci.getHost() +"/" + ci.getDatabase() + "?useCompression=true",
				ci.getUser(), 
				ci.getPassword()
		);
	}
}
\end{lstlisting}

\begin{lstlisting}[caption={ConnectionInformation (optimiert)}, label={lst:civ2}]
package de.whs.dbi.pa7.database;

/**
 * Dieses Model beinhaltet alle nötigen Informationen 
 * um sich zu einer Postgre Datenbank zu verbinden
 * 
 * @author Mario Kellner
 * @author Markus Hausmann
 * @author Jonas Stadtler
 *
 */
public class ConnectionInformation {
	
	// Eigenschaften
	public String host = null;
	public String database = null;
	public String user = null;
	public String password = null;
	
	
	// Getter & Setter
	public String getHost() {
		return host;
	}
	public void setHost(String host) {
		this.host = host;
	}
	public String getDatabase() {
		return database;
	}
	public void setDatabase(String database) {
		this.database = database;
	}
	public String getUser() {
		return user;
	}
	public void setUser(String user) {
		this.user = user;
	}
	public String getPassword() {
		return password;
	}
	public void setPassword(String password) {
		this.password = password;
	}
}
\end{lstlisting}

\begin{lstlisting}[caption={tpsCreator (optimiert)}, label={lst:tpsv2}]
package de.whs.dbi.pa7.database;

import java.io.ByteArrayInputStream;
import java.io.File;
import java.io.FileReader;
import java.io.FileWriter;
import java.io.IOException;
import java.io.StringReader;
import java.sql.SQLException;
import java.sql.Statement;
import java.util.concurrent.ThreadLocalRandom;

import org.postgresql.copy.CopyIn;
import org.postgresql.copy.CopyManager;
import org.postgresql.core.BaseConnection;

/**
 * Die tpsCreator Klasse enthält alle nötigen Befehle um eine initial n-tps-Datenbank
 * erstellen zu können. Hier kann auserdem ein instanziiertes Verbindungsobjekt übergeben werden,
 * um auf unterschiedliche Datenbanken gleichzeitig arbeiten zu können
 * 
 * @author Mario Kellner
 * @author Markus Hausmann
 * @author Jonas Stadtler
 */
public class TpsCreator implements TpsCreatorInterface {
	
	/**
	 * Debug Variable
	 */
	private boolean isDebug = false;
	
	/**
	 * Das Verbindungsobjekt
	 */
	private DatabaseConnection connection;
	
	/**
	 * Tabellen die in dem Shema enthalten sind, momentan nur nötig zum "droppen" der Datenbank
	 */
	public String[] tables = {"branches", "accounts", "tellers", "history"};
	
	/**
	 * Die Create-Statements für das Schema
	 */
	public String[] schema = {
			"CREATE TABLE branches (branchid INT NOT NULL, branchname CHAR(20) NOT NULL, balance INT NOT NULL, address CHAR(72) NOT NULL)",
			"CREATE TABLE accounts (accid INT NOT NULL, NAME CHAR(20) NOT NULL, balance INT NOT NULL, branchid INT NOT NULL, address CHAR(68) NOT NULL)",
			"CREATE TABLE tellers (tellerid INT NOT NULL, tellername CHAR(20) NOT NULL, balance INT NOT NULL, branchid INT NOT NULL, address CHAR(68) NOT NULL)",
			"CREATE TABLE history (accid INT NOT NULL, tellerid INT NOT NULL, delta INT NOT NULL, branchid INT NOT NULL, accbalance INT NOT NULL, cmmnt CHAR(30) NOT NULL)",
	};
	
	/**
	 * Alter Statements zum setzen der Keys
	 */
	public String[] AlterStatements = {
			"ALTER TABLE branches ADD PRIMARY KEY(branchid)",
			"ALTER TABLE accounts ADD PRIMARY KEY(accid)",
			"ALTER TABLE tellers ADD PRIMARY KEY(tellerid)",
			"ALTER TABLE accounts ADD FOREIGN KEY(branchid) REFERENCES branches(branchid)",
			"ALTER TABLE tellers ADD FOREIGN KEY(branchid) REFERENCES branches(branchid)",
			"ALTER TABLE history ADD FOREIGN KEY(branchid) REFERENCES branches(branchid)",
			"ALTER TABLE history ADD FOREIGN KEY(tellerid) REFERENCES tellers(tellerid)",
			"ALTER TABLE history ADD FOREIGN KEY(accid) REFERENCES accounts(accid)"
	};

	
	public String fileBranches;
	public String fileAccounts;
	public String fileTellers;
	
	/**
	 * Unser Konstruktor. Er Überprüft, ob ein nicht leeres DatabaseConnection Objekt übergeben worden ist.
	 * 
	 * 
	 * @param connection Das Verbindungsobjekt
	 * @throws NullPointerException Das Verbindungsobjekt ist leer 
	 */
	public TpsCreator(DatabaseConnection connection) throws Exception {
		System.out.println("Info: Optimierte Version wird benutzt!");
		
		if(connection == null ) {
			throw new NullPointerException("connection object cannot be null");
		}
		
		try {
			connection.databaseLink.isValid(0);
		}
		catch(Exception err) {
			throw err;
		}
		
		this.connection = connection;
		
	}

	/**
	 * Aktiviert das Transaktion-System.
	 */
	public void beginTransaktion() {
		try {
			connection.databaseLink.setAutoCommit(false);
		} catch (SQLException e) {
			e.printStackTrace();
		}
	}
	
	/**
	 * Comitted die Queries zur DAtenbank und beendet das Transaktion-System
	 * 
	 * @return Gibt an, ob win Fehler vorhanden ist.
	 */
	public void endAndCommitTransaction() {
		try {
			connection.databaseLink.commit();
			connection.databaseLink.setAutoCommit(true);
		} catch (SQLException e) {
			e.printStackTrace();
		}
	}
	
	/**
	 * Löscht die Tabellen aus der Datenbank
	 * 
	 * @return Gibt an, ob win Fehler vorhanden ist.
	 */
	public boolean dropSchema() {
		try {
			Statement statement = connection.databaseLink.createStatement();

			for (String table : this.tables) {
				StringBuilder strBuilder = new StringBuilder("DROP TABLE IF EXISTS ").append(table).append(" CASCADE");
				statement.executeUpdate(strBuilder.toString());
				
				if(isDebug) {
					System.out.println(strBuilder.toString());
				}
			}

			return true;
		} catch (SQLException e) {
			e.printStackTrace();
			return false;
		}
		
	}
	
	/**
	 * Erstellt die Tabellen
	 * 
	 * @return Gibt an, ob win Fehler vorhanden ist.
	 */
	public boolean createSchema() {
	
		try {
			Statement statement = connection.databaseLink.createStatement();

			for (String table : this.schema) {
				statement.executeUpdate(table);
				if(isDebug) {
					System.out.println(table);
				}
			}

			return true;
		} catch (SQLException e) {
			return false;
		}
	}
	
	/**
	 * Erstellt die Tabellen
	 * 
	 * @return Gibt an, ob win Fehler vorhanden ist.
	 */
	public boolean createKeys() {
	
		try {
			Statement statement = connection.databaseLink.createStatement();

			for (String table : this.AlterStatements) {
				statement.executeUpdate(table);
				if(isDebug) {
					System.out.println(table);
				}
			}

			return true;
		} catch (SQLException e) {
			System.out.println(e.getMessage());
			return false;
		}
	}
	
	/**
	 * Erstellt die Tupel in der Tabelle Branch.
	 * 
	 * @param n Anzahl der Tupel, die erstellt werden
	 * @return Gibt an, ob ein Fehler vorhanden ist
	 */
	public boolean createBranchTupel(int n) {
		try {
			
			CopyManager cpm = new CopyManager((BaseConnection) connection.databaseLink);
			long ci = cpm.copyIn("COPY branches (branchid, branchname, balance, address) FROM STDIN WITH DELIMITER '|'", new FileReader("branches.txt"));

			return true;
		} catch (SQLException e) {
			e.printStackTrace();
		} catch (IOException e) {
			// TODO Auto-generated catch block
			e.printStackTrace();
		}
		
		return false;
	}
	
	/**
	 * Erstellt die Tupel in der Tabelle Accounts
	 * 
	 * @param n Anzahl der Tupel, die erstellt werden mal 10000
	 * @return Gibt an, ob ein Fehler vorhanden ist
	 */
	public boolean createAccountTupel(int n) {
		try {
			CopyManager cpm = new CopyManager((BaseConnection) connection.databaseLink);
			long ci = cpm.copyIn("COPY accounts(accid, NAME, balance, address, branchid) FROM STDIN WITH DELIMITER '|'", new FileReader("accounts.txt"));
		
			return true;
		} catch (SQLException e) {
			e.printStackTrace();
		} catch (IOException e) {
			// TODO Auto-generated catch block
			e.printStackTrace();
		}
		
		return false;
	}	
	
	/**
	 * Erstellt die Tupel in der Tabelle Teller
	 * 
	 * @param n Anzahl der Tupel, die erstellt werden mal 10
	 * @return Gibt an, ob ein Fehler vorhanden ist
	 */
	public boolean createTellerTupel(int n) {		
		try {
			CopyManager cpm = new CopyManager((BaseConnection) connection.databaseLink);
			long ci = cpm.copyIn("COPY tellers (tellerid, tellername, balance, address, branchid) FROM STDIN WITH DELIMITER '|'", new FileReader("tellers.txt"));
			return true;
		} catch (SQLException e) {
			e.printStackTrace();
		} catch (IOException e) {
			// TODO Auto-generated catch block
			e.printStackTrace();
		}
		
		return false;
	}
	
	/**
	 * Bündelt die Funktionen des tpsCreators und führt diese nach ein ander aus.
	 * 
	 * @param n Anzahl der Tupel, die erstellt werden
	 */
	public void autoSetup(int n) {
		System.out.println("Starte automatisches Setup...");
		System.out.println("Erzeuge eine " + n + "-tps-Datenbank");
		
		if(!dropSchema()) {
			System.out.println("Warnung: Datenbank konnte nicht geleert werden!");
			return;
		}
		if(!createSchema()) {
			System.out.println("Tabellen konnten nicht erzeugt werden. Stoppe Setup ...");
			return;
		}
		
		if(!createBranchTupel(n)) {
			System.out.println("Branches konnten nicht angelegt werden. Stoppe Setup ...");
			return;
		}
		if(!createAccountTupel(n)) {
			System.out.println("Accounts konnten nicht angelegt werden. Stoppe Setup ...");
			return;
		}
		if(!createTellerTupel(n)) {
			System.out.println("Teller konnten nicht angelegt werden. Stoppe Setup ...");
			return;
		}

		System.out.println("Erstellen der " + n + "-tps-Datenbak erfolgreich");
	}

	/**
	 * Gibt an, ob das DebugLogging aktiviert ist.
	 */
	public boolean isDebug() {
		return isDebug;
	}
	
	/**
	 * Aktivert bzw deaktiviert den Debug Modus
	 */
	public void setDebug(boolean isDebug) {
		this.isDebug = isDebug;
	}

	@Override
	public void finishUp() {
		this.createKeys();
	}

	public void createFiles(int size) {
		System.out.println("Erstelle Benchmark Datein!");

		FileWriter fw1;
		try {
			System.out.print("Erstelle Branches ...");

			fw1 = new FileWriter(new File("branches.txt"));
			for (int i = 1; i <= size; i++) {
				
				fw1.write(i + "|aaaaaaaaaaaaaaaaaaaa|0|aaaaaaaaaaaaaaaaaaaaaaaaaaaaaaaaaaaaaaaaaaaaaaaaaaaaaaaaaaaaaaaaaaaaaaaa\n");
				
			}
			fw1.close();
			
			System.out.println("[OK]");
			System.out.print("Erstelle Accounts ...");			
			fw1 = new FileWriter(new File("accounts.txt"));
			for (int i = 1; i <= size*100000; i++) {

				 fw1.write(i + "|aaaaaaaaaaaaaaaaaaaa|0|aaaaaaaaaaaaaaaaaaaaaaaaaaaaaaaaaaaaaaaaaaaaaaaaaaaaaaaaaaaaaaaaaaaa|" + ThreadLocalRandom.current().nextInt(1, size + 1) + "\n");
			}
			
			fw1.close();

			System.out.println("[OK]");
			System.out.print("Erstelle Tellers ...");

			fw1 = new FileWriter(new File("tellers.txt"));
			for (int i = 1; i <= size*10; i++) {
				
				fw1.write(i +"|aaaaaaaaaaaaaaaaaaaa|0|aaaaaaaaaaaaaaaaaaaaaaaaaaaaaaaaaaaaaaaaaaaaaaaaaaaaaaaaaaaaaaaaaaaa|" +
						ThreadLocalRandom.current().nextInt(1, size + 1) +
				"\n");
			}
			
			fw1.close();
			System.out.println("[OK]");
			
		} catch (IOException e) {
			// TODO Auto-generated catch block
			e.printStackTrace();
		}
		
	}
}
\end{lstlisting}

\begin{lstlisting}[caption={TpsCreatorInterface (optimiert)}, label={lst:tpsiv2}]
package de.whs.dbi.pa7.database;

public interface TpsCreatorInterface {
	void beginTransaktion();
	void endAndCommitTransaction();
	boolean dropSchema();
	boolean createSchema();
	boolean createBranchTupel(int n);
	boolean createAccountTupel(int n);
	boolean createTellerTupel(int n);
	void autoSetup(int n);
	boolean isDebug();
	void setDebug(boolean isDebug);
	public void finishUp();
	public void createFiles(int size);
}
\end{lstlisting}

\begin{lstlisting}[caption={TpsCreatorOldStatements (optimiert)}, label={lst:tpsoldv2}]
package de.whs.dbi.pa7.database;

import java.sql.SQLException;
import java.sql.Statement;
import java.util.concurrent.ThreadLocalRandom;

/**
 * Die tpsCreator Klasse enthält alle nötigen Befehle um eine initial n-tps-Datenbank
 * erstellen zu können. Hier kann auserdem ein instanziiertes Verbindungsobjekt übergeben werden,
 * um auf unterschiedliche Datenbanken gleichzeitig arbeiten zu können
 * 
 * @author Mario Kellner
 * @author Markus Hausmann
 * @author Jonas Stadtler
 */
public class TpsCreatorOldStatements  implements TpsCreatorInterface {
	private boolean isDebug = false;
	
	/**
	 * Das Verbindungsobjekt
	 */
	private DatabaseConnection connection;
	
	/**
	 * Tabellen die in dem Shema enthalten sind, momentan nur nötig zum "droppen" der Datenbank
	 */
	public String[] tables = { "branches", "accounts", "tellers", "history"};
	
	/**
	 * Die Create-Statements für das Schema
	 */
	public String[] schema ={
			"CREATE TABLE branches (branchid INT NOT NULL, branchname CHAR(20) NOT NULL, balance INT NOT NULL, address CHAR(72) NOT NULL, PRIMARY KEY (branchid))",
			"CREATE TABLE accounts ( accid INT NOT NULL, NAME CHAR(20) NOT NULL, balance INT NOT NULL, branchid INT NOT NULL, address CHAR(68) NOT NULL, PRIMARY KEY (accid), FOREIGN KEY (branchid) REFERENCES branches )",
			"CREATE TABLE tellers ( tellerid INT NOT NULL, tellername CHAR(20) NOT NULL, balance INT NOT NULL, branchid INT NOT NULL, address CHAR(68) NOT NULL, PRIMARY KEY (tellerid), FOREIGN KEY (branchid) REFERENCES branches)",
			"CREATE TABLE history ( accid INT NOT NULL, tellerid INT NOT NULL, delta INT NOT NULL, branchid INT NOT NULL, accbalance INT NOT NULL, cmmnt CHAR(30) NOT NULL, FOREIGN KEY (accid) REFERENCES accounts, FOREIGN KEY (tellerid) REFERENCES tellers, FOREIGN KEY (branchid) REFERENCES branches )",
	};
	
	
	/**
	 * Unser Konstruktor. Er Überprüft, ob ein nicht leeres DatabaseConnection Objekt übergeben worden ist.
	 * 
	 * 
	 * @param connection Das Verbindungsobjekt
	 * @throws NullPointerException Das Verbindungsobjekt ist leer 
	 */
	public TpsCreatorOldStatements(DatabaseConnection connection) throws Exception {
		System.out.println("Info: Unoptimierte Version wird benutzt!");
		if(connection == null ) {
			throw new NullPointerException("connection object cannot be null");
		}
		
		try {
			connection.databaseLink.isValid(0);
			//connection.databaseLink.setAutoCommit(false);
		}
		catch(Exception err) {
			throw err;
		}
		
		this.connection = connection;
		
	}

	/**
	 * Aktiviert das Transaktion-System.
	 */
	public void beginTransaktion() {
		try {
			connection.databaseLink.setAutoCommit(false);
		} catch (SQLException e) {
			e.printStackTrace();
		}
	}
	
	/**
	 * Comitted die Queries zur DAtenbank und beendet das Transaktion-System
	 * 
	 * @return Gibt an, ob win Fehler vorhanden ist.
	 */
	public void endAndCommitTransaction() {
		try {
			connection.databaseLink.commit();
			connection.databaseLink.setAutoCommit(true);
		} catch (SQLException e) {
			e.printStackTrace();
		}
	}
	
	/**
	 * Löscht die Tabellen aus der Datenbank
	 * 
	 * @return Gibt an, ob ein Fehler vorhanden ist.
	 */
	public boolean dropSchema() {
		try {
			Statement statement = connection.databaseLink.createStatement();

			for (String table : this.tables) {
				statement.executeUpdate("DROP TABLE IF EXISTS " + table + " CASCADE");
				if(isDebug) {
					System.out.println("DROP TABLE IF EXISTS " + table + " CASCADE");
				}
			}

			return true;
		} catch (SQLException e) {
			e.printStackTrace();
			return false;
		}
		
	}
	
	/**
	 * Erstellt die Tabellen
	 * 
	 * @return Gibt an, ob win Fehler vorhanden ist.
	 */
	public boolean createSchema() {
	
		try {
			Statement statement = connection.databaseLink.createStatement();

			for (String table : this.schema) {
				statement.executeUpdate(table);
				if(isDebug) {
					System.out.println(table);
				}
			}

			return true;
		} catch (SQLException e) {
			return false;
		}
	}
	
	
	/**
	 * Erstellt die Tupel in der Tabelle Branch
	 * 
	 * @param n Anzahl der Tupel, die erstellt werden
	 * @return Gibt an, ob ein Fehler vorhanden ist
	 */
	public boolean createBranchTupel(int n) {
		try {
			Statement statement = connection.databaseLink.createStatement();

			for (int i = 1; i <= n; i++) {
				statement.executeUpdate("INSERT INTO branches (branchid, branchname, balance, address) VALUES(" + i + ", 'branch', 0, 'branch')");
				
				if(isDebug) {
					System.out.println("INSERT INTO branches (branchid, branchname, balance, address) VALUES(" + i + ", 'branch', 0, 'branch')");
				}
			}

			return true;
		} catch (SQLException e) {
			e.printStackTrace();
		}
		
		return false;
	}
	
	
	/**
	 * Erstellt die Tupel in der Tabelle Accounts
	 * 
	 * @param n Anzahl der Tupel, die erstellt werden mal 10000
	 * @return Gibt an, ob ein Fehler vorhanden ist
	 */
	public boolean createAccountTupel(int n) {
		int localConst = n*100000;
		int localRandom;
		
		try {
			
			Statement statement = connection.databaseLink.createStatement();

			for (int i = 1; i <= localConst; i++) {
				
				// Generiert eine zufällige ID zwischen 1 und n
				localRandom = ThreadLocalRandom.current().nextInt(1, n + 1);
								
				statement.executeUpdate("INSERT INTO accounts (accid, NAME, balance, address, branchid) VALUES("
					+ i + ", 'account', 0,'test', " + localRandom + ")");
		
				if(isDebug) {
					System.out.println("INSERT INTO accounts (accid, NAME, balance, address, branchid) "
							+ "VALUES(" + i + ", 'account', 0,'test', " + localRandom + ")");	
				}
			}

			return true;
		} catch (SQLException e) {
			e.printStackTrace();
		}
		
		return false;
	}
	
	/**
	 * Erstellt die Tupel in der Tabelle Teller
	 * 
	 * @param n Anzahl der Tupel, die erstellt werden mal 10
	 * @return Gibt an, ob ein Fehler vorhanden ist
	 */
	public boolean createTellerTupel(int n) {
		int localConst = n*10;
		int localRandom;
		
		try {
			
			Statement statement = connection.databaseLink.createStatement();

			for (int i = 1; i <= localConst; i++) {
				
				// Generiert eine zufällige ID zwischen 1 und n
				localRandom = ThreadLocalRandom.current().nextInt(1, n + 1);
								
				statement.executeUpdate("INSERT INTO tellers (tellerid, tellername, balance, address, branchid) "
						+ "VALUES(" + i + ", 'teller', 0,'adress', " + localRandom + ")");
				if(isDebug) {
					System.out.println("INSERT INTO tellers (tellerid, tellername, balance, address, branchid) "
							+ "VALUES(" + i + ", 'teller', 0,'adress', " + localRandom + ")");					
				}
			}

			return true;
		} catch (SQLException e) {

			e.printStackTrace();
		}
		
		return false;
	}
	
	
	/**
	 * Bündelt die Funktionen des tpsCreators und führt diese nach ein ander aus.
	 * 
	 * @param n Anzahl der Tupel, die erstellt werden
	 */
	public void autoSetup(int n) {
		System.out.println("Starte automatisches Setup...");
		System.out.println("Erzeuge eine " + n + "-tps-Datenbank");
		
		if(!dropSchema()) {
			System.out.println("Warnung: Datenbank konnte nicht geleert werden!");
			return;
		}
		if(!createSchema()) {
			System.out.println("Tabellen konnten nicht erzeugt werden. Stoppe Setup ...");
			return;
		}
		
		/**
		 * Jetzt kommen die Datensätze in die Datenbank yay \o/
		 */
		
		if(!createBranchTupel(n)) {
			System.out.println("Branches konnten nicht angelegt werden. Stoppe Setup ...");
			return;
		}
		if(!createAccountTupel(n)) {
			System.out.println("Accounts konnten nicht angelegt werden. Stoppe Setup ...");
			return;
		}
		if(!createTellerTupel(n)) {
			System.out.println("Teller konnten nicht angelegt werden. Stoppe Setup ...");
			return;
		}
		
		System.out.println("Erstellen der " + n + "-tps-Datenbak erfolgreich");
	}

	public boolean isDebug() {
		return isDebug;
	}

	public void setDebug(boolean isDebug) {
		this.isDebug = isDebug;
	}

	public void finishUp() {
		
	}
	public void createFiles(int size) {
		
	}
}
\end{lstlisting}
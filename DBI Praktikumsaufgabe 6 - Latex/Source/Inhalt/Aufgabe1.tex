\section{Praktkumsaufgabe 7)}
\subsection{Teilaufgabe A)}
\textbf{Entwickeln Sie ein Programm, das einen Aufrufparameter n erwartet und eine initiale n-
tps-Datenbank auf dem gewählten Datenbankmanagementsystem erzeugt.}

Um überhaupt mit der Datenbank zu interagieren mussten wir uns erstmal den
JDBC-Treiber für unser DBMS Postgresql besorgen.

Diesen haben wir anschließend in das Projekt eingebunden und haben mit der
Klasse \gqq{DatabaseConnection} diesen Treiber in der
\gqq{getConnection}-Methode laden lassen.


\lstset{language=Java, backgroundcolor=\color{editorGray},
  basicstyle={\linespread{0.82}\footnotesize\ttfamily},   
  frame=b, xleftmargin={0.75cm},literate=
    {Ö}{{\"O}}1
  {Ä}{{\"A}}1
  {Ü}{{\"U}}1
  {ß}{{\ss}}2
  {ü}{{\"u}}1
  {ä}{{\"a}}1
  {ö}{{\"o}}1}
\begin{lstlisting}
	/**
	 * List die Informatioen aus dem Objekt ci (ConnectionInformation) und versucht eine Verbindung 
	 * zur Datenbank auf zu bauen.
	 * 
	 * @throws SQLException Wird geworfen, wenn der DriverManager keine Verbndung zur Datenbank aufbauen kann
	 */
	public void connect() throws SQLException
	{
		/*
		 * Der Compiler uebersetzt " + var + " automatisch in ein StringBuilder Objekt
		 */
		databaseLink = DriverManager.getConnection(
				"jdbc:postgresql://" + ci.getHost() +"/" + ci.getDatabase(),
				ci.getUser(), 
				ci.getPassword()
		);
	}
\end{lstlisting}


Anschließend haben wir eine Klasse \gqq{tpsCreator} implementiert, welche die
gesamte Funktionalität zum Erstellen der \textbf{n-tps-Datenbank} beinhaltet.
Diese Klasse kann man wiederum ein \textit{DatabaseConnection}-Objekt übergeben.
So ist es Möglich in einem Programm mehrere die Datenbank auf verschiedene
Servern anzulegen. Die Verbindungsinformationen übergeben wir mit der Klasse
\nameref{lst:civ1}.

Der Quelltext von tpsCreator befindet sich im Anhang~\nameref{lst:tpsv1}

Als nächstes haben wir uns um die Benutzerinteraktion gekümmert. Wir waren uns
einig, dass wir anstatt auf eine komplexe GUI Darstellung auf eine simple
Konsolen-Anwendung beschränken wollen. Außerdem hat das große Vorteile gegenüer
der Performance.

Um die Interaktion mit der Konsole so einfach wie Möglich zu gestalten haben wir
uns eine Helfer Klasse ~\nameref{lst:crv1} geschrieben und diese anschließend
in unserer \nameref{lst:mainv1} benutzt.

\begin{lstlisting}
			System.out.println("Verbindungsinformationen eingeben!");
			
			System.out.println("Host:");
			infos.setHost(ConsoleReader.readString());

			System.out.println("Datenbank:");
			infos.setDatabase(ConsoleReader.readString());
			
			System.out.println("Benutzer:");
			infos.setUser(ConsoleReader.readString());
			
			System.out.println("Password:");
			infos.setPassword(ConsoleReader.readString());
\end{lstlisting}

Am Ende unserer Main rufen wir die Funkion autoSetup auf. Diese erstellt uns
anschliessend die Datenbank.

\subsection{Teilaufgabe B)}
\textbf{Welche Mindestgrößen schätzen Sie für eine 1-tps-Datenbank bzw. allgemein für eine n-
tps-Datenbank? Wie viel Plattenplatz wird auf dem Datenbank-Server tatsächlich für die
erstellten Datenbanken benötigt?}

Die reine Spaltenanzahl lässt sich folgendermaßen ausrechnen:
\begin{eqnarray}
n + n \cdot 10 + n \cdot 10000
\end{eqnarray}

Also sind bei einer 1-tps-Datenbank 10011 Datensätze enthalten.

Die physikalische Beträgt dabei: 17932460 Bytes. \newline
Umgerechnet ist die Datenbank dementsprechend 17 MB groß.

\subsection{Teilaufgabe C)}
\textbf{Versuchen Sie, die Laufzeit Ihres Programms zu beschleunigen! Dokumentieren Sie
einzelne Verbesserungsideen und die jeweiligen Laufzeitveränderungen für eine lokale
Ausführung Ihres Programms bei der Erzeugung einer 10-tps-Datenbank!}

Um den Durchsatz der Daten zu erhöhen lassen sich die Queries in einer
Transaction zusammenfassen.  

Dazu muss die Option \textit{autoCommit} abgeschaltet werden und die Transaktion
mit einem manuellen \textit{commit} zum Server geschickt werden. Wir müssen also
unseren \nameref{lst:tpsv1} erweitern.

-- bis hier her bin ich gekommmen. lg mario

\subsection{Teilaufgabe D)}
\textbf{Messen und protokollieren Sie die Laufzeit ihres optimierten Programms für n=10, n=20
und n=50 sowohl lokal auf dem Datenbank-Server als auch von einem "remote" Client!
(Gemessen werden soll nur die Laufzeit zum Einfügen ohne evtl. notwendige vorherige
DROP-TABLE- oder CREATE-TABLE-Anweisungen!)}


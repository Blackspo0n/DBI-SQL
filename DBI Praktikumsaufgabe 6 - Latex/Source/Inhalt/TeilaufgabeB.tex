\section{Teilaufgabe B)}
\textbf{Welche Mindestgrößen schätzen Sie für eine 1-tps-Datenbank bzw. allgemein für eine n-
tps-Datenbank? Wie viel Plattenplatz wird auf dem Datenbank-Server tatsächlich für die
erstellten Datenbanken benötigt?}

Die reine Spaltenanzahl lässt sich folgendermaßen ausrechnen:
\begin{eqnarray}
n + n \cdot 10 + n \cdot 100000 = \mbox{Spaltenanzahl}
\end{eqnarray}

Also sind bei einer 1-tps-Datenbank 100011 Datensätze enthalten. Plus den 4
Tabellen sind wir dann bei einer Anzahl von 100015 Statements.

Die Größe der Datenbank kann man in zwei Kategorien.
\begin{itemize}
  \item Relationale Größe
  \item Physikalische Größe
\end{itemize}

Folgender SQL-Query liefert uns die relationale-Größe jeder einzelner Tabelle:
\begin{lstlisting}[language=sql, caption={Größe der Tabelle und Datensätze
ermitteln}]
  SELECT nspname || '.' || relname AS relation,
    pg_total_relation_size(C.oid) AS total_size
  FROM pg_class C
  LEFT JOIN pg_namespace N ON (N.oid = C.relnamespace)
  WHERE nspname NOT IN ('pg_catalog', 'information_schema')
    AND C.relkind <> 'i'
    AND nspname !~ '^pg_toast'
  ORDER BY pg_total_relation_size(C.oid) DESC
\end{lstlisting}

Wenn wir diesen Query ausführen erhalten wir folgende Ausgabe:
\tabelle{Ausgabe des Query}{tab:databaseSize}{databaseSize.tex}

Die so errechnete Größe liegt bei 16441344 bytes. Umgerechnet ergibt das: 
15,68 MB.

Die tatsächliche physikalische Größe der gesamten Datenbank lässt sich mit
folgendem Query bestimmen:
\begin{lstlisting}[language=sql, caption={Physikalische Größe der Datenbank
ermitteln}]
SELECT d.datname AS Name,  pg_catalog.pg_get_userbyid(d.datdba) AS Owner, pg_catalog.pg_database_size(d.datname) AS SIZE FROM pg_catalog.pg_database d
    ORDER BY pg_catalog.pg_database_size(d.datname) DESC
\end{lstlisting}

Die tatsächliche Größe der Datenbank beträgt dabei: 24469676 Bytes. Umgerechnet
ist die Datenbank dementsprechend 23,34 MB groß.

Die Differenz beider Größen beträgt also 7,66 MB.

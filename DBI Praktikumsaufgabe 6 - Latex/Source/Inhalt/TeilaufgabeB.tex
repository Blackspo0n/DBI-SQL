\section{Teilaufgabe B)}
\textbf{Welche Mindestgrößen schätzen Sie für eine 1-tps-Datenbank bzw. allgemein für eine n-
tps-Datenbank? Wie viel Plattenplatz wird auf dem Datenbank-Server tatsächlich für die
erstellten Datenbanken benötigt?}

Die reine Spaltenanzahl lässt sich folgendermaßen ausrechnen:
\begin{eqnarray}
n + n \cdot 10 + n \cdot 10000 = \mbox{Spaltenanzahl}
\end{eqnarray}

Also sind bei einer 1-tps-Datenbank 10011 Datensätze enthalten.


Folgender SQL-Query liefert uns die physikalische Größe jeder einzelner Tabelle:
\begin{lstlisting}[language=sql]
  SELECT nspname || '.' || relname AS relation,
    pg_total_relation_size(C.oid) AS total_size
  FROM pg_class C
  LEFT JOIN pg_namespace N ON (N.oid = C.relnamespace)
  WHERE nspname NOT IN ('pg_catalog', 'information_schema')
    AND C.relkind <> 'i'
    AND nspname !~ '^pg_toast'
  ORDER BY pg_total_relation_size(C.oid) DESC
\end{lstlisting}
Wenn wir diesen Query ausführen erhalten wir folgende Ausgabe:
\tabelle{Ausgabe des Query}{tab:databaseSize}{databaseSize.tex}

Die so errechnete Größe liegt bei 1736704 bytes. Umgerechnet ergibt das: 
1,65 MB.

Die tatsächliche physikalische Größe der gesammten Datenbank lässt sich mit
folgendem Query bestimmen:
\begin{lstlisting}[language=sql]
SELECT d.datname AS Name,  pg_catalog.pg_get_userbyid(d.datdba) AS Owner,
    pg_catalog.pg_database_size(d.datname) AS SIZE
FROM pg_catalog.pg_database d
    ORDER BY pg_catalog.pg_database_size(d.datname) DESC
\end{lstlisting}

Die tatsäcliche Größe der Datenban beträgt dabei: 17932460 Bytes. Umgerechnet
ist die Datenbank dementsprechend 17,1 MB groß.

Die Differenz beträgt also 15,45 MB.

--- hier noch eine begründung warum das so ist.
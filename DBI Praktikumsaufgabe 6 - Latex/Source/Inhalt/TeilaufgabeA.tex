\section{Teilaufgabe A)}
\textbf{Entwickeln Sie ein Programm, das einen Aufrufparameter n erwartet und eine initiale n-
tps-Datenbank auf dem gewählten Datenbankmanagementsystem erzeugt.}

Um überhaupt eine Verbindung zu unserem DBMS\footnote{Datenbank Magement
System} PostgreSQL verbinden zu können, müssen wir uns erstmal den JDBC-Treiber
für unser DBMS besorgen. Der JDBC-Treiber befindet sich auf der
Webseite von PostgreSQL\footnote{\url{https://jdbc.postgresql.org/download.html}}.

Diesen haben wir anschließend in das Projekt eingebunden und haben mit der
Klasse \gqq{DatabaseConnection} diesen Treiber in der
\gqq{getConnection}-Methode laden lassen.

\lstset{language=Java, backgroundcolor=\color{editorGray},
  basicstyle={\linespread{0.82}\footnotesize\ttfamily},   
  frame=b, xleftmargin={0.75cm},literate=
    {Ö}{{\"O}}1
  {Ä}{{\"A}}1
  {Ü}{{\"U}}1
  {ß}{{\ss}}2
  {ü}{{\"u}}1
  {ä}{{\"a}}1
  {ö}{{\"o}}1,
    numberstyle=\tiny\noncopynumber,
      }
\begin{lstlisting}[caption={connect Funktion}]
	/**
	 * List die Informatioen aus dem Objekt ci (ConnectionInformation) und versucht eine Verbindung 
	 * zur Datenbank auf zu bauen.
	 * 
	 * @throws SQLException Wird geworfen, wenn der DriverManager keine Verbndung zur Datenbank aufbauen kann
	 */
	public void connect() throws SQLException
	{
		databaseLink = DriverManager.getConnection(
				"jdbc:postgresql://" + ci.getHost() +"/" + ci.getDatabase(),
				ci.getUser(), 
				ci.getPassword()
		);
	}
\end{lstlisting}

Die Verbindungsinformationen übergeben wir mittels der Klasse
\hyperref[lst:civ2]{ConnectionInformation}. Diese implementiert ein einfaches
Model, welches nur die Parameter für die Datenbankverbindung beinhaltet.

Anschließend implementieren wir eine Klasse \hyperref[lst:tpsv2]{TpsCreator},
welche die gesamte Funktionalität zum Erstellen der \textbf{n-tps-Datenbank} beinhaltet.
Dieser Klasse kann man wiederum ein
\hyperref[lst:dbv2]{DatabaseConnection}-Objekt übergeben. Durch diese Handhabung
ist es möglich in einem Programm Datenbanken auf verschiedene Servern anzulegen.

Die Queries der Klasse \hyperref[lst:tpsv2]{TpsCreator}
führen wir mit anschließend normalen Statements aus. 

\begin{lstlisting}[caption={Eine der drei Tupel Funktionen}]
	/**
	 * Erstellt die Tupel in der Tabelle Branch
	 * 
	 * @param n Anzahl der Tupel, die erstellt werden
	 * @return Gibt an, ob ein Fehler vorhanden ist
	 */
	public boolean createBranchTupel(int n) {
		try {
			Statement statement = connection.databaseLink.createStatement();

			for (int i = 1; i <= n; i++) {
				statement.executeUpdate("INSERT INTO branches (branchid, branchname, balance, address) VALUES(" + i + ", 'branch', 0, 'branch')");
			}
			return true;
		} catch (SQLException e) {}		
		return false;
	}
\end{lstlisting}

Als nächstes kümmern wir uns um die Benutzerinteraktion. Wir waren uns
einig, dass wir uns anstelle einer komplexen GUI Darstellung auf eine simple
Konsolen-Anwendung beschränken wollen. Denn diese Form der
Nutzerinteraktion hat große Vorteile gegenüber der Performance, da wir nicht
das Swing-Framework laden und initialisieren müssen und uns so Wertvolle
Sekunden im Benchmarking sparen können.

Um die Interaktion mit der Konsole so einfach wie möglich zu gestalten,
schreiben wir uns eine Helfer Klasse \hyperref[lst:crv2]{ConsoleReader} und
benutzen diese anschließend in unserer \hyperref[lst:mainv2]{Main}-Klasse.

\begin{lstlisting}[caption={Verbindungsinformationen per Konsole}]
System.out.println("Verbindungsinformationen eingeben!");
			
System.out.println("Host:");
infos.setHost(ConsoleReader.readString());

System.out.println("Datenbank:");
infos.setDatabase(ConsoleReader.readString());

System.out.println("Benutzer:");
infos.setUser(ConsoleReader.readString());

System.out.println("Password:");
infos.setPassword(ConsoleReader.readString());
\end{lstlisting}

Am Ende unserer \hyperref[lst:mainv2]{Main}-Klasse rufen wir die Funkion
\textit{autoSetup} in dem initialisieren \hyperref[lst:tpsv2]{TpsCreator}-Objekt
auf. Diese Funktion erstellt uns anschließend die Datenbank. Ein Durchlauf mit
einer \textbf{10-tps-Datenbank} dauert \ca 4 Minuten und ist alles andere als performant.
\clearpage

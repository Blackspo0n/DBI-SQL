\section{Teilaufgabe A)}
\textbf{Entwickeln Sie ein Programm, das einen Aufrufparameter n erwartet und eine initiale n-
tps-Datenbank auf dem gewählten Datenbankmanagementsystem erzeugt.}

Um überhaupt mit unserer PostgreSQL Datenbank zu können interagieren mussten wir
uns erstmal den JDBC-Treiber für unser DBMS Postgresql besorgen. Den
JDBC-Treiber befindet sich au der Webseite von
PostgreSQL\footnote{\url{https://jdbc.postgresql.org/download.html}}.

Diesen haben wir anschließend in das Projekt eingebunden und haben mit der
Klasse \gqq{DatabaseConnection} diesen Treiber in der
\gqq{getConnection}-Methode laden lassen.

\lstset{language=Java, backgroundcolor=\color{editorGray},
  basicstyle={\linespread{0.82}\footnotesize\ttfamily},   
  frame=b, xleftmargin={0.75cm},literate=
    {Ö}{{\"O}}1
  {Ä}{{\"A}}1
  {Ü}{{\"U}}1
  {ß}{{\ss}}2
  {ü}{{\"u}}1
  {ä}{{\"a}}1
  {ö}{{\"o}}1,
    numberstyle=\tiny\noncopynumber,
      }
\begin{lstlisting}
	/**
	 * List die Informatioen aus dem Objekt ci (ConnectionInformation) und versucht eine Verbindung 
	 * zur Datenbank auf zu bauen.
	 * 
	 * @throws SQLException Wird geworfen, wenn der DriverManager keine Verbndung zur Datenbank aufbauen kann
	 */
	public void connect() throws SQLException
	{
		/*
		 * Der Compiler uebersetzt " + var + " automatisch in ein StringBuilder Objekt
		 */
		databaseLink = DriverManager.getConnection(
				"jdbc:postgresql://" + ci.getHost() +"/" + ci.getDatabase(),
				ci.getUser(), 
				ci.getPassword()
		);
	}
\end{lstlisting}

Die Verbindungsinformationen übergeben wir mittels der Klasse
\nameref{lst:civ1}. Diese implementiert ein einfaches Model, welches nur die
Parameter für die Datenbankverbindung beinhaltet.

Anschließend haben wir eine Klasse \nameref{lst:tpsv1} implementiert, welche die
gesamte Funktionalität zum Erstellen der \textbf{n-tps-Datenbank} beinhaltet.
Dieser Klasse kann man wiederum ein \nameref{lst:dbv1}-Objekt
übergeben. So ist es möglich in einem Programm die Datenbank auf
verschiedene Servern anzulegen. 

Die Queries der Klasse \nameref{lst:tpsv1}
führen wir mit anschließend normalen Statements aus. 

\begin{lstlisting}
	/**
	 * Erstellt die Tupel in der Tabelle Branch
	 * 
	 * @param n Anzahl der Tupel, die erstellt werden
	 * @return Gibt an, ob ein Fehler vorhanden ist
	 */
	public boolean createBranchTupel(int n) {
		try {
			Statement statement = connection.databaseLink.createStatement();

			for (int i = 1; i <= n; i++) {
				statement.executeUpdate("INSERT INTO branches (branchid, branchname, balance, address) VALUES(" + i + ", 'branch', 0, 'branch')");
			}
			return true;
		} catch (SQLException e) {}		
		return false;
	}
\end{lstlisting}

Als nächstes haben wir uns um die Benutzerinteraktion gekümmert. Wir waren uns
einig, dass wir uns anstelle einer komplexen GUI Darstellung auf eine simple
Konsolen-Anwendung beschränken wollen. Außerdem hat das große Vorteile gegenüber
der Performance, da wir das Swing-Framework nicht laden initialisieren
brauchen.

Um die Interaktion mit der Konsole so einfach wie möglich zu gestalten haben wir
uns eine Helfer Klasse ~\nameref{lst:crv1} geschrieben und diese anschließend
in unserer \nameref{lst:mainv1} benutzt.

\begin{lstlisting}
			System.out.println("Verbindungsinformationen eingeben!");
			
			System.out.println("Host:");
			infos.setHost(ConsoleReader.readString());

			System.out.println("Datenbank:");
			infos.setDatabase(ConsoleReader.readString());
			
			System.out.println("Benutzer:");
			infos.setUser(ConsoleReader.readString());
			
			System.out.println("Password:");
			infos.setPassword(ConsoleReader.readString());
\end{lstlisting}

Am Ende unserer \nameref{lst:mainv1}-Klasse rufen wir die Funkion
\textit{autoSetup} in dem instanziierten \nameref{lst:tpsv1}-Objekt auf. Diese Funktion erstellt uns
anschliessend die Datenbank. Ein Durchlauf mit einer \textbf{10-tps-Datenbank}
dauert \ca 4 Minuten und ist alles andere als performant.
\clearpage
